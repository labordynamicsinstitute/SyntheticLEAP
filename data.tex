
In this section, we briefly describe the structure of the three data sources.% before summarizing the steps taken to harmonize the sources and discussing the limitations in the current application.

\subsection{United States: \acf{LBD}}

The \ac{LBD} \citep{LBD} is created from the U.S. Census Bureau's \ac{BR} by creating longitudinal links of establishments using name and address matching. The database has information on birth, death, location, industry,  firm  affiliation of employer establishments, and ownership by multi-establishment firms, as well as their employment over time, for nearly all sectors of the economy from 1976 through 2015 (as of this writing). It serves as a key linkage file as well as a research dataset in its own right for numerous research articles, as well as a tabulation input to the U.S. Census Bureau's \acl{BDS} \citep[\acs{BDS}]{BDS}. Other statistics created from the underlying Business Register include the \acl{CBP} \citep[\acs{CBP}]{CBP} and the \acl{SBUSB} \citep[\acs{SBUSB}]{SBUSB}. For a full description, readers should consult  \citet{RePEc:cen:wpaper:02-17,}. The key variables of interest for this experiment are birth and death dates, payroll, employment, and the industry coding of the establishment.  \citet{SJIAOS-2014d} explore a possible expansion of the synthesis methods described later to include location and firm affiliation. Note that information on payroll and employment does not come from individual-level wage records, as is the case for both the Canadian and German datasets described below, as well as for the \acl{QWI} \citep{AbowdEtAl2009} derived from the \acl{LEHD} \citep[LEHD]{RePEc:cen:wpaper:18-27} in the United States. Thus, methods that connect establishments based on labor flows \citep{BenedettoEtAl2007,RePEc:iab:iabfme:201006_en} are not employed. We also note that payroll is the cumulative sum of wages paid over the entire calendar year, whereas employment is measured as of March 12 of each year.


%The US LBD was created in the early 2000s (Miranda and Jarmin, 2002), following previous research files with more restrictive coverage. At its core, it is a research database containing longitudinally linked data records from a statistical business register of establishments. Breaks of longitudinal links are resolved using probabilistic name and address matching. The variables currently in the LBD are industry, annual payroll, employment, geography, birth year, death year, and firm structure. Though it has very few variables on the database itself, it serves as a backbone for many linkages into establishment and firm surveys and censuses such as business dynamics and job flows. 

%The fundamental structure of the LBD (and thus the SynLBD) is a longitudinal file on economic entities, where each entity has a start and end date and a small number of key attributes that evolve over time. Hypothetically, this structure is shared by many other longitudinal panels, such as panels of jobs or of residences. We should note that it does not apply to data structures like a linked employer-employee database, since there are no linkages between entities at a point in time. Thus, using concepts from graph theory, it is a mapping of a network that contains only nodes, and no edges. These structural characteristics are relevant for any attempt to generalize the synthesizing methodology to other contexts, such as matched employer-employee data (but see Barrientos et al., 2017).

%The primary goal of the SynLBD project is to create partially synthetic microdata on establishments for public release, allowing researchers easier access for the implementation of a wide (unconstrained) range of models with analytically valid inferences about the underlying population, while protecting against re-identification of any given unit or its attributes. There are multiple reasons why a public release of such data is desirable. The US LBD is one of most requested datasets in the Federal Statistical Research Data Centers (FSRDCs), but access through FSRDC the is still subject to long approval processes. In many European countries, access to data on business registers is arduous or impossible for researchers. Access through commercial providers is possible (Bureau Van Dijk), but coverage is generally poor.




\subsection{Canada: \acf{LEAP}}


The \ac{LEAP} contains information on annual employment for each employer business in all sectors of the Canadian economy. It covers incorporated and unincorporated businesses that issue at least one annual statement of remuneration paid (T4 slips) in any given calendar year. It excludes self-employed individuals or partnerships with non-salaried participants.

To construct the LEAP, Statistics Canada uses three sources of information: (1) T4 administrative data  from the Canada Revenue Agency (CRA), (2) data from Statistics Canada's \ac{BR}, and (3) data from  Statistics Canada's Survey of Employment, Payrolls and Hours (SEPH). 
\begin{description}
\item[T4] In general, all employers in Canada need to fill out a T4 slip to submit to the CRA if they paid employment income, taxable allowances and benefits, or any other remuneration in any calendar year. 

\item[BR] The Business Register is Statistics Canada's central repository of baseline information on business and institutions operating in Canada. It is used as the survey frame for all business related data sets.

\item[SEPH] The objective of the SEPH is to provide monthly information on the level of earnings, the number of jobs, and hours worked by detailed industry at the national and provincial levels. To do so, it combines a census of approximately one million payroll deductions provided by the CRA, and the Business Payrolls Survey, a sample of 15,000 establishments.  
\end{description}
The LEAP essentially contains four variables (1) A Longitudinal Business Register Identifier (LBRID), (2) Industry, (3) Employment and (4) Payroll. 
\begin{enumerate}

\item The LBRID uniquely identifies each enterprise and is derived from the Business Register. To avoid ``false'' deaths and births due to mergers, restructuring or changes in reporting practices, Statistics Canada uses a method of ``labour tracking'' that compares cluster of workers in each newly identified enterprise with all the clusters of workers in firms from the previous year. This comparison yields a new identifier (LBRID) derived from those of the BR.

\item The industry information comes from the BR for single-industry firms. If a firm operates in multiple industries, information on payroll from the SEPH is used to identify the industry in which the firm pays the highest payroll. Prior to 1991, information on industry was based on the SIC but it is now based on the  North American Industrial Classification System (NAICS) at the four-digits level. 

\item Employment is measured either using Individual Labour Unit (ILU) or Average Labour Unit (ALU). ALUs are obtained by dividing the business annual payroll (as the sum of T4 slips income issued in the year) and diving by the average annual earnings in its industry/province/class category computed using the SEPH. ILUs is a head count of the number of T4 issued by the enterprise, with employees working for multiple employers split proportionately across firms according to their total annual payroll earned in each firm. 

\item Finally, the firm's payroll comes from the sum of all T4s as reported to the CRA.

\end{enumerate}
With that information, the LEAP is the only data set in Canada  that allows research on a variety of themes, like employment growth, industry turnover, firm survival, job creation and job destruction, etc. 


\subsection{Germany: \acf{BHP}}

%Because the Institute for Employment Research is affiliated with the German Ministry of Labour and Social Affairs, it collects very little data at the business or establishment level.\footnote{Exceptions are the IAB Establishment Panel \citep{IABEstabPanel} and the IAB Job Vacancy Survey \citep{JVS}.}

The core database for the Establishment History Panel is the German Social Security Data  (GSSD), which is based on the integrated notification procedure for the health, pension and unemployment insurances,   introduced in  1973. Employers report information on all their employees, by establishment. Aggregating this information via an establishment identifier yields the Establishment History Panel \citep[German abbreviation: BHP]{BHP}. We used data from  1975 until 2008, which at the time this project started was the most current data available for research. Information for the former Eastern German States is limited to the years 1992-2008. 

Due to the purpose and structure of the GSSD, some variables present in the \ac{LBD} are not available on the  \ac{BHP}. Firm-level information is not captured, and it is thus not known whether establishments are part of a multi-establishment employer. In 1999, reporting requirements were extended to all establishments; prior to that date, only establishments that  had at least one employee covered by social security on the reference date June 30 of each year were subject to filing requirements. Payroll and employment are both based on a reference date of June 30, and are thus consistent point-in-time measures. 
Industries are identified according to the WZ 2008 classification system \citep{WZ2008} on the five digit level.\footnote{The WZ 2008 classification system is compliant with the requirements of the Statistical Classification of Economic Activities in the European Community (NACE Rev. 2).} However, we aggregated the industry information for this project by only using the first four digits of the coding system.






\subsection{Harmonizing and Preprocessing}

In all countries, the underlying data provide annual measures. However, \SynLBD{} assumes a longitudinal (wide) structure of the dataset, with invariant industry (and location). In all cases, the modal industry is chosen to represent the entity's industrial activity. 
%
Further adjustments made to the \ac{BHP} for this project include estimating full-year payroll, creating time-consistent geographic information, and applying employment flow methods \citep{RePEc:iab:iabfme:201006_en} to adjust for spurious births and deaths in establishment identifiers. \citet{SJIAOS-2014b} provide a detailed description of the steps taken to harmonize the input data. 


In both Canada and Germany, we encountered various technical and data-driven limitations. In all countries, data in the first year and last year are occasionally problematic, and such data  were dropped. 
%
Both the German and the Canadian data experience some level of industry coding change, which may affect the classification of some entities. Furthermore, due to the nature of the underlying data, entities are establishments in Germany and the US, but employers in Canada. 

After the various standardizations and choices made above, the data structure is intended to be comparable, as summarized in Table~\ref{tab:common_Variable}. The column "Nature" identifies the treatment of the variable in the synthesis process \SynLBD. 

% Table tab:LEAP_Variable
%\begin{table}[H]
  \centering\footnotesize
  \caption{CanSynLEAP variable descriptions}  \label{tab:LEAP_Variable} \medskip
  \renewcommand{\arraystretch}{1}
  \begin{tabular}{l  c c c c c}
    \toprule
    \textbf{Name}&\textbf{Type}&\textbf{Description}&\textbf{Notation}&\textbf{Action}\\
    \midrule
synid&Identifier&Unique random number for enterprise&&Created\\
NAICS&Categorical&4 digit industry code&$x_{1}$&Unmodified\\
Firstyear&Categorical&First year enterprise is observed &$y_{1}$&Synthesized\\
Lastyear&Categorical&Last year enterprise is observed &$y_{2}$&Synthesized\\
Year&Categorical&Year dating of annual variables&&Created\\
ALU&Continuous&(annual)&$y_{3}$&Synthesized\\
Payroll&Continuous&Payroll (annual)&$x_{4}$&Synthesized\\
   \bottomrule
  \end{tabular} 
\\
Note: Variables denoted with $y_{i}$ are synthesized and variables denoted with $x_{i}$ are not synthesized. 
\end{table}
\begin{table}[H]
  \centering\footnotesize
  \caption{Variable descriptions and comparison}  \label{tab:common_Variable} \medskip
  \renewcommand{\arraystretch}{1}
  \begin{tabular}{l  c c c c c c c}
    \toprule
    \textbf{Name}&\textbf{Type} &\textbf{Description} &\textbf{US} & \textbf{Canada} &\textbf{Germany} &\textbf{Nature}\\
    \midrule
Entity Identifier& identifier& & Establishment & Employer & Establishment &Created\\
Industry code&Categorical& Various across countries &SIC3 & NAICS4 & WZ93 &Unmodified\\
             &           &                          &(3-digit )& (4-digit) &  &\\
First year&Categorical&First year entity is observed &\multicolumn{3}{c}{--- firstyear ---}&Synthesized\\
Last year&Categorical&Last year entity is observed &\multicolumn{3}{c}{--- lastyear ---}&Synthesized\\
Year&Categorical&Year dating of annual variables&\multicolumn{3}{c}{--- year ---}&Derived\\
Employment & Continuous & Employment measure & Count & ALU & Count & Synthesized \\
            &            &                    & (March 15) &Average Labor Unit (annual)& (June 30)&\\
Payroll&Continuous&  Payroll (annual)& Measured & Computed & Adjusted &Synthesized\\
   \bottomrule
  \end{tabular} 
%\\
%Note: Variables denoted with $y_{i}$ are synthesized and variables denoted with $x_{i}$ are not synthesized. 
\end{table}
