%Because the Institute for Employment Research is affiliated with the German Ministry of Labour and Social Affairs, it collects very little data at the business or establishment level.\footnote{Exceptions are the IAB Establishment Panel \citep{IABEstabPanel} and the IAB Job Vacancy Survey \citep{JVS}.}

The core database for the Establishment History Panel is the German Social Security Data  (GSSD), which is based on the integrated notification procedure for the health, pension and unemployment insurances,   introduced in  1973. Employers report information on all their employees, by establishment. Aggregating this information via an establishment identifier yields the Establishment History Panel \citep[German abbreviation: BHP]{BHP}. We used data from  1975 until 2008, which at the time this project started was the most current data available for research. Information for the former Eastern German States is limited to the years 1992-2008. 

Due to the purpose and structure of the GSSD, some variables present in the \ac{LBD} are not available on the  \ac{BHP}. Firm-level information is not captured, and it is thus not known whether establishments are part of a multi-establishment employer. In 1999, reporting requirements were extended to all establishments; prior to that date, only establishments that  had at least one employee covered by social security on the reference date June 30 of each year were subject to filing requirements. Payroll and employment are both based on a reference date of June 30, and are thus consistent point-in-time measures. 
Industries are identified according to the WZ 2003 classification system \citep{WZ2008} on the five digit level.\footnote{The WZ 2003 classification system is compliant with the requirements of the Statistical Classification of Economic Activities in the European Community (NACE Rev. 2).} However, we aggregated the industry information for this project by only using the first four digits of the coding system.




