
%Statistics Canada disseminates business data in highly aggregated forms. To get access to Canadian micro business databases, 
In this paper, we presented results from two projects that evaluated whether code developed to synthesize the U.S. LBD can easily be adapted to create synthetic versions of similar data from Canada and Germany. We considered both univariate time-series comparisons as well as model-based comparisons of coefficients and model fit. In general, utility evaluations show significant differences between each country's synthetic and confidential data. Frequently-used measures such as confidence interval overlap and $pMSE$ suggest that the synthetic data are an unreliable image of the confidential data. Less formal comparisons of specification test scores suggest that the synthetic data do not reliably lead to  the same modeling decisions.

Interestingly, the utility of the German synthetic data was higher than the utility of the Canadian data in almost all dimensions evaluated. At this point we can only speculate about potential reasons. The most important difference between the two data sources is that the German data comprises only a handful of industries while almost all industries have been included in the Canadian evaluation. Given that the industries included in the German data were rather large, and synthesis models are run independently for each industry, it might have been easier to preserve the industry level statistics for the German data. We cannot exclude the possibility that  the structure of the German data aligns more closely with the LBD and thus the synthesis models tuned on the LBD data provide better results on the (adjusted) BHP than on the LEAP. We note that both the LBD and the BHP are establishment-level datasets, whereas the LEAP is a employer-level dataset. 

We emphasize that adjustments to the original synthesis code were explicitly limited to ensuring that the code runs on the new input data. The validity of the synthetic data could possibly be improved by tuning the synthesis models to the particularities of the data at hand, such as the non-standard dynamics introduced into the German data by reunification.  However, the aim of this project was to illustrate that the high investments necessary for developing the synthesis code for the LBD offered additional payoffs as the re-use of the code substantially reduced the amount of work required to generate decent synthetic data products for other business data. One of the major criticisms of the synthetic data approach has been  that investments necessary to develop useful synthesizers are substantial. This project illustrated that substantial gains can be achieved when exploiting knowledge from previous projects. With the advent of tailor-made software such as the \textit{synthpop} package in R \citep{JSSv074i11}, the investments for generating useful synthetic data might be further reduced in the future.

However, even without  fine-tuning or customization of models, the current synthetic data have, in fact, proven useful. De facto, many deployments of synthetic data, including the Synthetic LBD in the US, have been used for model preparation by researchers in a public or lower-security environment, with subsequent remote submission of prepared code for validation against the confidential data. When viewed through the lens of such a validation system, the synthetic data prepared here would seem to have reasonable utility. While time series dynamics are not the same, they are broadly similar. Models converged in similar fashions, and while coefficients were strictly different, they were broadly similar and plausible. Specification tests did not lead to the same conclusions, but they also did not collapse or yield meaningless conclusions. Thus, we believe that the synthetic data, despite being different, have the potential to be a useful tool for analysts to prepare models without direct access to the confidential data. \textcite{VilhuberAbowd2016-SOLE,Vilhuber2019-SGP} come to a similar conclusion when evaluating usage of the synthetic datasets available through the Synthetic Data Server \citep{AbowdVilhuber2010}, including the Synthetic LBD. A more thorough evaluation would need to explicitly measure the investment in synthetic data generation, the cost of setting up a validation structure, and the number of studies enabled through such a setup. We note that such an evaluation is non-trivial: the counter-factual in many circumstances is that no access is allowed to sensitive business microdata, or that access occurs through a secure research data system that is also costly to maintain.
 
The use of synthetic datasets to broaden access to confidential microdata is likely to increase in the near future, with increasing concerns by statistical agencies regarding the disclosure risks of releasing microdata. The resulting reduction in access to scientific microdata is overwhelmingly seen as problematic. Broadly ``plausible'' if not analytically valid synthetic datasets such as those described in this paper, combined with scalable remote submission systems that integrate modern disclosure avoidance mechanisms, may be a feasible mitigation strategy. 
 
%The synthetic datasets used for this paper have not been released to a broader public. More substantial evaluations of any remaining disclosure risk would be necessary before an actual release of the data might happen. 




