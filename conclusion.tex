
%Statistics Canada disseminates business data in highly aggregated forms. To get access to Canadian micro business databases, 
In this paper, we presented results from two projects that evaluated whether code developed to synthesize the U.S. LBD can easily be adapted to synthesize similar data from Canada and Germany. Overall, the results look promising. While our utility evaluations find differences between the original data and the synthetic data, general patterns such as trends over time are preserved. Considering that most analyst would use the synthetic data only for data preparation and to develop their models, which would then be run on the original data, the data seems (mostly) fit for purpose at least along those dimensions, which have been evaluated in this paper. Obviously, more substantial evaluations would be necessary before an actual release of the data. Interestingly, the utility of the German synthetic data was higher than the utility of the Canadian data in almost all dimensions evaluated. At this point we can only speculate about potential reasons. The most important difference between the two data sources is that the German data comprises only a handful of industries while almost all industries have been included in the Canadian evaluation. Given that the industries included in the German data were rather large, and synthesis models are run independently for each industry, it might have been easier to preserve the industry level statistics for the German data. It might also be the case that the structure of the German data aligns more closely with the LBD and thus the synthesis models tuned on the LBD data provide better results on the (adjusted) BHP than on the LEAP. 

We emphasize that adjustments to the original synthesis code were limited to ensuring that the code runs on the new input data. The validity of the synthetic data could certainly be improved by fine tuning the synthesis models to the data at hand for example by specifically accounting for the reunification in Germany. However, the aim of this project was to illustrate that the high investments necessary for developing the synthesis code for the LBD offered additional payoffs as the re-use of the code substantially reduced the amount of work required to generate decent synthetic data products for other business data. As one of the major criticisms of the synthetic data approach is the substantial investments necessary to develop useful synthesizers, this project illustrated that substantial gains can be achieved when exploiting knowledge from previous projects. With the advent of tailor-made software such as the \textit{synthpop} package in R, the investments for generating useful synthetic data might be reduced further in the future.
