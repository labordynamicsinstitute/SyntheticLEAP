
\newcommand{\CanTableNote}{$LEAP$ is the Longitudinal Employment Analysis Program and $CanSynLBD$ is the Canadian synthetic database based on LEAP. }

\subsection{Entity Characteristics}

The CanSynLBD and LEAP generally provide comparable inferences on aggregate means and correlations. For example, Figures \ref{tab:Can:GrossEmploymentPrivate} and \ref{tab:Can:GrossEmploymentManufacturing} show that gross employment levels for each year in the CanSynLBD are very close to those in the LEAP. However, the manufacturing sector shows closer patterns than the private sector.\footnote{The private sector comprises all industries including the manufacturing sector except the public sector  (NAICS 61, 62, and 91)} We find similar results for total payroll (Figures \ref{tab:Can:TotalPayrollPrivate} and  \ref{tab:Can:TotalPayrollManufacturing}) .

\todo{Why is manufacturing always below, but overall employment crosses? Which industries are driving that?} \todo{JA: I checked before, but I could not able to identify any specific reason.}
\begin{figure} [H]
\centering
\caption{Gross employment level by year (private)} \label{tab:Can:GrossEmploymentPrivate}
\includegraphics[height=2.8in, width=.7\linewidth]{graphs/Gross_employment_level_by_year_private_bw.pdf} 
\begin{minipage}{0.85\textwidth}
{\footnotesize Note: \CanTableNote \par}
\end{minipage}
\end{figure}

\begin{figure}
\begin{subfigure}[h]{0.48\linewidth}
\includegraphics[width=\linewidth]{graphs/Gross_employment_level_by_year_private_bw.pdf}
\caption{CanSynLBD}
\end{subfigure}
\hfill
\begin{subfigure}[h]{0.48\linewidth}
\includegraphics[width=\linewidth]{graphs/Gross_employment_level_by_year_bw_GsynLBD.pdf}
\caption{GSynLBD}
\end{subfigure}%
\caption{Gross employment level by year}
\end{figure}



\begin{figure} [H]
\centering
\caption{Gross employment level by year (manufacturing)} \label{tab:Can:GrossEmploymentManufacturing}
\includegraphics[height=2.8in, width=.7\linewidth]{graphs/Gross_employment_level_by_year_manufacturing_bw.pdf} 
\begin{minipage}{0.85\textwidth}
{\footnotesize Note: \CanTableNote  \par}
\end{minipage}
\end{figure}


\begin{figure} [H]
\centering
\caption{Total payroll by year (private)} \label{tab:Can:TotalPayrollPrivate}
\includegraphics[height=2.8in, width=.7\linewidth]{graphs/Total_payroll_by_year_private_bw.pdf} 
\begin{minipage}{0.85\textwidth}
{\footnotesize Note: \CanTableNote \par}
\end{minipage}
\end{figure}
\begin{figure} [H]
\centering
\caption{Total payroll by year (manufacturing)} \label{tab:Can:TotalPayrollManufacturing}
\includegraphics[height=2.8in, width=.7\linewidth]{graphs/Total_payroll_by_year_manufacturing_bw.pdf} 
\begin{minipage}{0.85\textwidth}
{\footnotesize Note: \CanTableNote \par}
\end{minipage}
\end{figure}

\subsection{Entity Dynamics}
To assess how well the synthetic data capture entity dynamics, we also compute entry and exit rates  by year. Table \ref{tab:Can:FirmDynamics} shows that those rates for CanSynLBD are similar to LEAP database. To show further those rates are similar, we compute the divergence of entry rate as the entry rate of CanSynLBD net the entry rate of LEAP as well as the divergence of exit rate as the exit rate of CanSynLBD net the exit rate of LEAP (see Figure \ref{Divergence}).

\begin{table}[H]
  \centering
\begin{threeparttable}
 \caption{Entry and exit rates by year (LEAP)} \label{tab:Can:FirmDynamics} \medskip
\renewcommand{\arraystretch}{1}
\begin{tabular}{l|c c| c c| c c}
\toprule
&\multicolumn{2}{c|}{\textbf{LEAP}} &  \multicolumn{2}{c|}{\textbf{CanSynLBD}}&  \multicolumn{2}{c}{\textbf{Divergence}}\\
\textbf{Year}&\textbf{Entry Rate}&\textbf{Exit Rate}&\textbf{Entry Rate}&\textbf{Exit Rate} &\textbf{Entry Rate}&\textbf{Exit Rate}\\
\midrule
1992&11.77&11.72&11.16&11.71&-0.60&-0.00\\
1993&11.81&11.61&10.84&12.18&-0.97&0.57\\
1994&12.04&11.79&11.57&12.01&-0.47&0.22\\
1995&11.94&12.09&11.69&12.26&-0.25&0.17\\
1996&12.91&10.31&12.62&10.64&-0.29&0.32\\
1997&13.18&9.75&13.03&10.21&-0.15&0.47\\
1998&12.48&10.89&12.97&10.13&0.50&-0.75\\
1999&12.00&10.66&12.16&9.97&0.16&-0.69\\
2000&11.80&10.51&11.59&9.70&-0.20&-0.82\\
2001&11.44&10.20&11.33&9.52&-0.12&-0.68\\
2002&11.39&9.91&11.10&9.03&-0.29&-0.89\\
2003&11.17&10.21&10.52&9.37&-0.65&-0.84\\
2004&12.13&9.76&10.94&9.57&-1.20&-0.20\\
2005&11.92&10.07&11.07&9.86&-0.84&-0.21\\
2006&11.81&9.96&11.15&9.34&-0.66&-0.62\\
2007&12.28&9.80&10.99&9.31&-1.29&-0.49\\
2008&11.60&10.14&10.78&9.75&-0.82&-0.40\\
2009&10.77&9.93&9.99&9.81&-0.78&-0.12\\
2010&10.80&9.75&9.91&9.65&-0.89&-0.10\\
2011&10.62&9.79&9.73&10.00&-0.89&0.21\\
2012&10.60&9.76&10.02&10.20&-0.58&0.44\\
2013&10.16&9.71&9.95&10.32&-0.21&0.62\\
2014&9.93&10.11&9.26&10.70&-0.67&0.59\\

   \bottomrule
  \end{tabular} 
\begin{tablenotes}
\small
\item Note: \CanTableNote  We calculate the divergence of entry rate as the entry rate of CanSynLBD net of the entry rate of LEAP and the divergence of exit rate as the exit rate of CanSynLBD net of the exit rate of LEAP. Private sector only.
 \end{tablenotes}
 \end{threeparttable}
\end{table}

\begin{table}[H]
  \centering
\begin{threeparttable}
 \caption{Entry and exit rates by year (GLBD)} \label{tab:DE:FirmDynamics} \medskip
\renewcommand{\arraystretch}{1}
\begin{tabular}{l|c c| c c| c c}
\toprule
&\multicolumn{2}{c|}{\textbf{GLBD}} &  \multicolumn{2}{c|}{\textbf{GSynLBD}}&  \multicolumn{2}{c}{\textbf{Divergence}}\\
\textbf{Year}&\textbf{Entry Rate}&\textbf{Exit Rate}&\textbf{Entry Rate}&\textbf{Exit Rate} &\textbf{Entry Rate}&\textbf{Exit Rate}\\
\midrule
1977&10.09&9.25&8.06&5.17&-2.03&-4.08\\
1978&10.81&9.22&7.97&5.40&-2.84&-3.81\\
1979&11.20&8.37&8.38&5.26&-2.81&-3.11\\
1980&10.85&8.77&7.74&5.40&-3.11&-3.37\\
1981&9.88&9.36&6.68&6.22&-3.20&-3.14\\
1982&9.15&9.63&6.48&5.83&-2.67&-3.80\\
1983&9.19&9.31&6.35&5.52&-2.84&-3.80\\
1984&10.51&8.58&7.10&5.08&-3.41&-3.50\\
1985&9.74&9.82&7.48&5.81&-2.26&-4.01\\
1986&12.01&9.08&8.04&5.34&-3.97&-3.74\\
1987&11.10&8.46&8.18&5.74&-2.92&-2.72\\
1988&11.15&9.51&7.64&5.36&-3.50&-4.15\\
1989&11.21&8.75&8.26&5.92&-2.95&-2.83\\
1990&13.11&9.00&10.44&5.85&-2.68&-3.15\\
1991&14.37&9.20&19.51&5.70&5.14&-3.50\\
1992&12.90&11.07&16.66&6.91&3.76&-4.17\\
1993&27.66&9.38&10.44&7.81&-17.23&-1.57\\
1994&13.13&11.68&10.34&7.54&-2.78&-4.14\\
1995&13.36&11.18&9.79&7.44&-3.57&-3.74\\
1996&12.15&11.39&8.46&7.46&-3.69&-3.93\\
1997&12.33&11.34&9.76&6.96&-2.57&-4.39\\
1998&14.59&11.30&10.58&7.34&-4.01&-3.96\\
1999&22.29&9.85&15.22&7.65&-7.07&-2.20\\
2000&12.92&11.52&9.07&9.42&-3.85&-2.10\\
2001&11.22&12.52&8.22&9.47&-3.00&-3.05\\
2002&10.63&12.99&7.41&8.93&-3.21&-4.06\\
2003&11.19&12.47&7.62&8.89&-3.57&-3.59\\
2004&11.54&10.71&7.82&7.95&-3.72&-2.76\\
2005&10.23&11.07&7.20&7.95&-3.03&-3.12\\
2006&10.23&10.25&7.28&7.15&-2.94&-3.10\\
2007&9.82&8.88&6.05&7.47&-3.77&-1.41\\
2008&8.73&9.61&5.93&8.24&-2.80&-1.37\\

   \bottomrule
  \end{tabular} 
\begin{tablenotes}
\small
\item Note: We calculate the divergence of entry rate as the entry rate of GSynLBD net of the entry rate of GLBD and the divergence of exit rate as the exit rate of GSynLBD net of the exit rate of GLBD.
 \end{tablenotes}
 \end{threeparttable}
\end{table}

\begin{figure} [H]
\centering
\caption{Divergence of exit and entry rate between LEAP and CanSynLBD} \label{Divergence}
\includegraphics[height=2.8in, width=.7\linewidth]{graphs/Divergence_of_exit_and_entry_rate_between_LEAP_and_CanSynLBD_bw.pdf} 
\begin{minipage}{0.85\textwidth}
{\footnotesize Note: \CanTableNote  We calculate the divergence of entry rate as the entry rate of CanSynLBD net the entry rate of LEAP and the divergence of exit rate as the exit rate of CanSynLBD net the exit rate of LEAP. \par}
\end{minipage}
\end{figure}

\subsection{Dynamics of Job Flows}

One of the most important applications of LEAP is to generate statistics that describe job flows. Following \cite{DavisHaltiwangerSchuh}, the job creation is defined as the sum of all employment gains from expanding firms from year $t-1$ to year $t$ including entry firms. The job destruction rate is defined as the sum of all employment losses from contracting firms from year $t-1$ to year $t$ including exiting firms. Net job creation is the job creation rate minus the job destruction rate. Figures \ref{JobCreationPrivate} and \ref{JobCreationManufacturing} show the job creation rates from the CanSynLBD compared againg those of the LEAP. These figures show that the manufacturing sector has closer pattern than the private sector. We find a similar patterns for net job creation rates (Figures \ref{NetJobCreationPrivate} and  \ref{NetJobCreationManufacturing}).

\begin{figure} [H]
\centering
\caption{Job creation rate by year (private)} \label{JobCreationPrivate}
\includegraphics[height=2.8in, width=.7\linewidth]{graphs/Job_creation_rate_by_year_private_bw.pdf} 
\begin{minipage}{0.85\textwidth}
{\footnotesize Note: \CanTableNote \par}
\end{minipage}
\end{figure}

\begin{figure} [H]
\centering
\caption{Job creation rate  by year (manufacturing)} \label{JobCreationManufacturing}
\includegraphics[height=2.8in, width=.7\linewidth]{graphs/Job_creation_rate_by_year_Manufacturing_bw.pdf} 
\begin{minipage}{0.85\textwidth}
{\footnotesize Note: \CanTableNote \par}
\end{minipage}
\end{figure}

\todo{LV regraph, dropping last year} \todo{JA: Should we mention this in the text including reasons if we drop the last year here?}
\begin{figure} [H]
\centering
\caption{Net job creation rate by year (private)} \label{NetJobCreationPrivate}
\includegraphics[height=2.8in, width=.7\linewidth]{graphs/Net_job_creation_rate_by_year_private_bw.pdf} 
\begin{minipage}{0.85\textwidth}
{\footnotesize Note: \CanTableNote \par}
\end{minipage}
\end{figure}
\begin{figure} [H]
\centering
\caption{Net job creation rate  by year (manufacturing)} \label{NetJobCreationManufacturing}
\includegraphics[height=2.8in, width=.7\linewidth]{graphs/Net_job_creation_rate_by_year_Manufacturing_bw.pdf} 
\begin{minipage}{0.85\textwidth}
{\footnotesize Note: $LEAP$ is the Longitudinal Employment Analysis Program and $CanSynLBD$ is the Canadian synthetic database based on LEAP. In this graph, we use 2015 vintage of LEAP for the manufacturing sector and drop last year observation of each firm. \par}
\end{minipage}
\end{figure}





\subsection{Distribution of variables across time and industry}

\SynLBD{} keeps fixed the total number of entities that ever exist within the time frame used, but because each entity's entry and exit date are synthesized, the total number of entities at a particular point in time may differ, and with it employment and payroll. We inspect this distribution by industry in the following figures. 


Figures~\ref{FirmSharePrivate} and \ref{FirmShareManufacturing} plot the share of firms by two-digit industry and year for both the Canadian synthetic  and confidential data. If only contemporaneous features (employment, payroll), and birth and death of the synthetic entities were the same, all observations would be on the 45 degree line. The figures show some divergence of the within-industry distribution across time. 

\begin{figure} [H]
\centering
\caption{Share of firms by NAICS two-digit and year (private)} \label{FirmSharePrivate}
\includegraphics[height=2.8in, width=.7\linewidth]{graphs/Share_of_firms_by_NAICS_two-digit_and_year_private_bw.pdf} 
\begin{minipage}{0.85\textwidth}
{\footnotesize Note: \CanTableNote \par}
\end{minipage}
\end{figure}


\vspace{-15.5pt}
\begin{figure} [H]
\centering
\caption{Share of firms by NAICS two-digit and year (manufacturing)} \label{FirmShareManufacturing}
\includegraphics[height=2.8in, width=.7\linewidth]{graphs/Share_of_firms_by_NAICS_two-digit_and_year_Manufacturing_bw.pdf} 
\begin{minipage}{0.85\textwidth}
{\footnotesize Note: \CanTableNote \par}
\end{minipage}
\end{figure}

Given a distribution of entities alive, we can also compute the share of observed characteristics across time and industry. Shares of a variable $X$ are computed as a fraction of the sum of $X$ across \textit{all} years and industries:

\begin{equation}
    \label{eq:share_employment}
x_{its} = X_{its}/\sum_{i} \sum_{t} X_{its}, 
\end{equation}

where $i$ are two-digit NAICS industries, $t$ are  years in-sample, $s$ indicates whether it is in the synthetic or confidential data, and $X_{its}$ is the sum of the variable of interest $X$ for industry $i$ and year $t$ in  dataset $s$.

Figures~\ref{EmploymentSharePrivate} and \ref{EmploymentShareManufacturing} plot the share of employment by two-digit industry and year for both  CanSynLBD and the LEAP database. 
Employment shares  do not cluster along the 45-degree line. However, this hides significant differences between sectors. For instance,  the share of employment for the manufacturing sector does show stronger clustering along the 45-degree line.

\begin{figure} [H]
\centering
\caption{Share of employment by NAICS two-digit and year (private)} \label{EmploymentSharePrivate}
\includegraphics[height=2.8in, width=.7\linewidth]{graphs/Share_of_employment_by_NAICS_two-digit_and_year_private_bw.pdf} 
\begin{minipage}{0.85\textwidth}
{\footnotesize Note: \CanTableNote \par}
\end{minipage}
\end{figure}
\vspace{-15.5pt}
\begin{figure} [H]
\centering
\caption{Share of employment by NAICS two-digit and year (manufacturing)} \label{EmploymentShareManufacturing}
\includegraphics[height=2.8in, width=.7\linewidth]{graphs/Share_of_employment_by_NAICS_two-digit_and_year_Manufacturing_bw.pdf} 
\begin{minipage}{0.85\textwidth}
{\footnotesize Note: \CanTableNote \par}
\end{minipage}
\end{figure}

Figures~\ref{PayrollSharePrivate} and~\ref{PayrollShareManufacturing} plot the share of \textit{payroll} by two-digit industry and year for both CanSynLBD and LEAP database. In general, shares do not cluster along the 45-degree line, though a focus on  the manufacturing sector again shows stronger clustering along the 45-degree line.

\begin{figure} [H]
\centering
\caption{Share of payroll by NAICS two-digit and year (private)} \label{PayrollSharePrivate}
\includegraphics[height=2.8in, width=.7\linewidth]{graphs/Share_of_payroll_by_NAICS_two-digit_and_year_private_bw.pdf} 
\begin{minipage}{0.85\textwidth}
{\footnotesize Note: \CanTableNote \par}
\end{minipage}
\end{figure}
\vspace{-15.5pt}
\begin{figure} [H]
\centering
\caption{Share of payroll by NAICS two-digit and year (manufacturing)} \label{PayrollShareManufacturing}
\includegraphics[height=2.8in, width=.7\linewidth]{graphs/Share_of_payroll_by_NAICS_two-digit_and_year_Manufacturing_bw.pdf} 
\begin{minipage}{0.85\textwidth}
{\footnotesize Note: \CanTableNote \par}
\end{minipage}
\end{figure}

\subsection{pMSE}


To compute the $pMSE$, we estimate equation \ref{pMSE} using both the logit and probit models. TTable~\ref{tab:pMSE_regression} shows the results from the estimation of $pMSE$ for the Canadian data.
\todo{BD: Are those coefficients or marginal effects? Does it make sense to show coefficients? Or are we interested only in the last row? If we are interested in the coefficient, why is there no discussion of those?} 
\todo{Those are coefficients. I think we are interested in the last row.} 
$pMSE$ is closer to zero for the manufacturing sector than the private sector in both regressions, consistent with our earlier observations for Canada.

%\newpage
\begin{table}[H]
  \centering
\begin{threeparttable}
 \caption{$pMSE$ estimates for CanSynLBD} \label{tab:pMSE_regression} \medskip
\renewcommand{\arraystretch}{1}
\begin{tabular}{l|c c| c c}
\toprule
\textbf{Independent Variables}&\multicolumn{2}{c|}{\textbf{Logistic Regression}} &  \multicolumn{2}{c}{\textbf{Probit Regression}}\\
\midrule
          &\multicolumn{1}{c}{Manufacturing}&\multicolumn{1}{c}{Private}&\multicolumn{1}{c}{Manufacturing}&\multicolumn{1}{c}{Private}\\
\hline
Ln ALU    &   0.1580\sym{***}&   0.7138\sym{***}&   0.1003\sym{***}&   0.4390\sym{***}\\
          & (0.0039)         & (0.0010)         & (0.0024)         & (0.0006)         \\
[1em]
Ln Pay    &   0.0039         &  -0.4426\sym{***}&   0.0012         &  -0.2691\sym{***}\\
          & (0.0037)         & (0.0010)         & (0.0023)         & (0.0006)         \\
[1em]
Age 3-4   &   0.0392\sym{***}&   0.0972\sym{***}&   0.0252\sym{***}&   0.0618\sym{***}\\
          & (0.0078)         & (0.0017)         & (0.0049)         & (0.0010)         \\
[1em]
Age 5-7   &  -0.0382\sym{***}&   0.0477\sym{***}&  -0.0233\sym{***}&   0.0309\sym{***}\\
          & (0.0073)         & (0.0016)         & (0.0045)         & (0.0010)         \\
[1em]
Age 8-12  &  -0.1258\sym{***}&  -0.0263\sym{***}&  -0.0781\sym{***}&  -0.0152\sym{***}\\
          & (0.0071)         & (0.0015)         & (0.0044)         & (0.0009)         \\
[1em]
Age 13 or more&  -0.2190\sym{***}&  -0.1024\sym{***}&  -0.1365\sym{***}&  -0.0627\sym{***}\\
          & (0.0074)         & (0.0016)         & (0.0046)         & (0.0010)         \\
\hline
\(N\)     &  2243011         & 34638723         &  2243011         & 34638723         \\
pseudo \(R^{2}\)&   0.0112         &   0.0318         &   0.0112         &   0.0320         \\
pMSE      &   0.0041         &   0.0121         &   0.0041         &   0.0124         \\

   \bottomrule
  \end{tabular} 
\begin{tablenotes}
\small
\item Note: An observation is a entity-year in the combined database. In all specifications, we include both time and industry fixed effects. Standard errors are in parentheses.  ***, **, and * indicate statistically significant coefficients at 1\%, 5\%, and 10\% percent levels, respectively.
 \end{tablenotes}
 \end{threeparttable}
\end{table}

\begin{table}[H]
  \centering
%\begin{threeparttable}
 \caption{$pMSE$ estimates for GSynLBD} \label{tab:pMSE_regression} \medskip
\renewcommand{\arraystretch}{1}
\begin{tabular}{l|c |c}
\toprule
\textbf{Independent Variables}&\textbf{Logistic Regression} &\textbf{Probit Regression}\\
\midrule
Ln ALU    &  -0.2895\sym{***}&  -0.1812\sym{***}\\
          & (0.0033)         & (0.0021)         \\
[1em]
Ln Pay    &   0.2584\sym{***}&   0.1618\sym{***}\\
          & (0.0028)         & (0.0018)         \\
[1em]
Age 3-4   &  -0.0987\sym{***}&  -0.0616\sym{***}\\
          & (0.0070)         & (0.0043)         \\
[1em]
Age 5-7   &  -0.0973\sym{***}&  -0.0608\sym{***}\\
          & (0.0066)         & (0.0041)         \\
[1em]
Age 8-12  &  -0.1172\sym{***}&  -0.0733\sym{***}\\
          & (0.0063)         & (0.0039)         \\
[1em]
Age 13 or more&  -0.1487\sym{***}&  -0.0930\sym{***}\\
          & (0.0059)         & (0.0037)         \\
\hline
\(N\)     &  2121956         &  2121956         \\
pseudo \(R^{2}\)&   0.0038         &   0.0038         \\
pMSE      &   0.0013         &   0.0013         \\

   \bottomrule
  \end{tabular} 
\begin{tablenotes}
\small
\item Note: An observation is a entity-year in the combined database. In all specifications, we include both time and industry fixed effects. Standard errors are in parentheses.  ***, **, and * indicate statistically significant coefficients at 1\%, 5\%, and 10\% percent levels, respectively.
 \end{tablenotes}
% \end{threeparttable}
\end{table}

DISCUSSION IS STILL MISSING.\todo{pMSE discussion}

\subsection{Regression Analysis}

To assess how well the synthetic data perform in a more complex model, we estimate  the following dynamic panel data model:
\begin{eqnarray}	
Emp_{et} & = & \alpha + \theta Emp_{e,t-1} + \lambda Pay_{et} + Age_{et}^{T}\beta + \lambda_t + \alpha_i + \epsilon_{et}
\end{eqnarray}
where $Emp_{et}$ is log employment of entity $e$ in year $t$, $Emp_{e,t-1}$ is its one year lag, $Pay_{et}$ is the logarithm of payroll of entity $e$ in year $t$, $Age_{et}$ is a vector of dummy variables for age of entity $e$ in year $t$, $\lambda_t$ is the year fixed effect, $\alpha_i$ is an unobserved time-invariant industry-specific effect, and $\epsilon_{et}$ is the disturbance term of entity $e$ in year $t$. This allows us to assess whether the synthetic data capture variability in economic growth due to industry and firm age.

\begin{table}[H]
  \centering
\begin{threeparttable}
 \caption{Regression coefficients (OLS) for LEAP} \label{OLS} \medskip
\renewcommand{\arraystretch}{1}
\begin{tabular}{l|c c| c c}
\toprule
\textbf{Independent Variables}&\multicolumn{2}{c|}{\textbf{LEAP}} &  \multicolumn{2}{c}{\textbf{CanSynLBD}}\\
\midrule
&\multicolumn{1}{c}{Private}&\multicolumn{1}{c}{Manufacturing}&\multicolumn{1}{c}{Private}&\multicolumn{1}{c}{Manufacturing}\\
\hline
AR(1) Coefficient&   0.2031\sym{***}&   0.2481\sym{***}&   0.3970\sym{***}&   0.4405\sym{***}\\
          & (0.0001)         & (0.0005)         & (0.0002)         & (0.0007)         \\
[1em]
Ln Pay    &   0.7847\sym{***}&   0.7300\sym{***}&   0.5481\sym{***}&   0.5228\sym{***}\\
          & (0.0001)         & (0.0005)         & (0.0002)         & (0.0006)         \\
[1em]
Age 3-4   &  -0.1202\sym{***}&  -0.1717\sym{***}&  -0.1223\sym{***}&  -0.2340\sym{***}\\
          & (0.0003)         & (0.0014)         & (0.0004)         & (0.0016)         \\
[1em]
Age 5-7   &  -0.1260\sym{***}&  -0.1891\sym{***}&  -0.1235\sym{***}&  -0.2507\sym{***}\\
          & (0.0003)         & (0.0014)         & (0.0004)         & (0.0016)         \\
[1em]
Age 8-12  &  -0.1268\sym{***}&  -0.1973\sym{***}&  -0.1169\sym{***}&  -0.2551\sym{***}\\
          & (0.0003)         & (0.0013)         & (0.0004)         & (0.0016)         \\
[1em]
Age 13 or more&  -0.1246\sym{***}&  -0.1992\sym{***}&  -0.1101\sym{***}&  -0.2577\sym{***}\\
          & (0.0003)         & (0.0014)         & (0.0004)         & (0.0017)         \\
\hline
\(N\)     & 15708195         &  1015293         & 13573225         &   959764         \\
\(R^{2}\) &   0.9696         &   0.9743         &   0.9444         &   0.9523         \\

   \bottomrule
  \end{tabular} 
\begin{tablenotes}
\small
\item Note: In all specifications, we include both year and industry fixed effects. Standard errors are in parentheses.  ***, **, and * indicate statistically significant coefficients at 1\%, 5\%, and 10\% percent levels, respectively.
 \end{tablenotes}
 \end{threeparttable}
\end{table}

\begin{table}[H]
  \centering
%\begin{threeparttable}
 \caption{Regression coefficients (OLS) for GLBD} \label{OLS} \medskip
\renewcommand{\arraystretch}{1}
\begin{tabular}{l|c |c}
\toprule
\textbf{Independent Variables}&\textbf{GLBD} &  \textbf{GSynLBD}\\
\midrule

AR(1) Coefficient&   0.4430\sym{***}&   0.4143\sym{***}\\
          & (0.0007)         & (0.0008)         \\
[1em]
Ln Pay    &   0.4629\sym{***}&   0.5143\sym{***}\\
          & (0.0006)         & (0.0007)         \\
[1em]
Age 3-4   &  -0.0695\sym{***}&  -0.0642\sym{***}\\
          & (0.0017)         & (0.0016)         \\
[1em]
Age 5-7   &  -0.1066\sym{***}&  -0.0891\sym{***}\\
          & (0.0017)         & (0.0016)         \\
[1em]
Age 8-12  &  -0.1324\sym{***}&  -0.1109\sym{***}\\
          & (0.0017)         & (0.0016)         \\
[1em]
Age 13 or more&  -0.1880\sym{***}&  -0.1600\sym{***}\\
          & (0.0016)         & (0.0015)         \\
\midrule
\(N\)     &   848871         &   966084         \\
\(R^{2}\) &   0.9167         &   0.8968         \\

   \bottomrule
  \end{tabular} 
\begin{tablenotes}
\small
\item Note: In all specifications, we include both year and industry fixed effects. Standard errors are in parentheses.  ***, **, and * indicate statistically significant coefficients at 1\%, 5\%, and 10\% percent levels, respectively.
 \end{tablenotes}
 %\end{threeparttable}
\end{table}

\todo{adjust for german analysis}
We estimate the model separately on confidential and synthetic data for the private sector (and for Canada, for the manufacturing sector). We find that the CansynLBD data provides similar predictions to LEAP data (Tables  \ref{OLS}).\todo{compute overlap interval} \todo{JA: @Lars, I think you mentioned once that you would like to calculate this. I could calculate using the method explained in the appendix.}

\begin{table}[H]
  \centering
\begin{threeparttable}
 \caption{Regression coefficients (Dynamic) for LEAP} \label{Dynamic - GMM} \medskip
\renewcommand{\arraystretch}{1}
\begin{tabular}{l|c c| c c}
\toprule
\textbf{Independent Variables}&\multicolumn{2}{c|}{\textbf{LEAP}} &  \multicolumn{2}{c}{\textbf{CanSynLBD}}\\
\midrule
&\multicolumn{1}{c}{Private}&\multicolumn{1}{c}{Manufacturing}&\multicolumn{1}{c}{Private}&\multicolumn{1}{c}{Manufacturing}\\
\hline
AR(1) Coefficient&   0.0805\sym{***}&   0.1189\sym{***}&   0.5722\sym{***}&   0.5425\sym{***}\\
          & (0.0003)         & (0.0018)         & (0.0024)         & (0.0084)         \\
[1em]
Ln Pay    &   0.8991\sym{***}&   0.8523\sym{***}&   0.4101\sym{***}&   0.4302\sym{***}\\
          & (0.0002)         & (0.0015)         & (0.0018)         & (0.0067)         \\
[1em]
Age 3-4   &  -0.0450\sym{***}&  -0.0797\sym{***}&  -0.2075\sym{***}&  -0.2972\sym{***}\\
          & (0.0002)         & (0.0014)         & (0.0010)         & (0.0051)         \\
[1em]
Age 5-7   &  -0.0438\sym{***}&  -0.0860\sym{***}&  -0.2129\sym{***}&  -0.3162\sym{***}\\
          & (0.0002)         & (0.0015)         & (0.0011)         & (0.0059)         \\
[1em]
Age 8-12  &  -0.0418\sym{***}&  -0.0923\sym{***}&  -0.2187\sym{***}&  -0.3294\sym{***}\\
          & (0.0003)         & (0.0017)         & (0.0013)         & (0.0070)         \\
[1em]
Age 13 or more&  -0.0379\sym{***}&  -0.0898\sym{***}&  -0.2318\sym{***}&  -0.3414\sym{***}\\
          & (0.0003)         & (0.0019)         & (0.0015)         & (0.0080)         \\
\hline
\(N\)     & 15708195         &  1015293         & 13573225         &   959764         \\
m2        & -14.5000         &  -2.2200         & -27.5400         &  -9.4400         \\
Sargan test&  6.9e+04         &  4.6e+03         &  1.5e+04         &  1.5e+03         \\
df of Sargan Test& 252.0000         & 252.0000         & 252.0000         & 252.0000         \\
P value of Sargan test&   0.0000         &   0.0000         &   0.0000         &   0.0000         \\

   \bottomrule
  \end{tabular} 
\begin{tablenotes}
\small
\item Note: In this table, $m2$ is the Arellano-Bond test for zero autocorrelation in first-differenced errors for order two. Standard errors are in parentheses. ***, **, and * indicate statistically significant coefficients at 1\%, 5\%, and 10\% percent levels, respectively.
 \end{tablenotes}
 \end{threeparttable}
\end{table}

\begin{table}[H]
  \centering
%\begin{threeparttable}
 \caption{Regression coefficients (Dynamic) for GLBD} \label{Dynamic - GMM} \medskip
\renewcommand{\arraystretch}{1}
\begin{tabular}{l|c |c}
\toprule
\textbf{Independent Variables}&\textbf{GLBD} &\textbf{GSynLBD}\\
\midrule
AR(1) Coefficient&   0.0489\sym{***}&   0.6999\sym{***}\\
          & (0.0051)         & (0.0057)         \\
[1em]
Ln Pay    &   0.7559\sym{***}&   0.2916\sym{***}\\
          & (0.0035)         & (0.0042)         \\
[1em]
Age 3-4   &  -0.0070\sym{***}&  -0.1026\sym{***}\\
          & (0.0012)         & (0.0015)         \\
[1em]
Age 5-7   &  -0.0233\sym{***}&  -0.1386\sym{***}\\
          & (0.0014)         & (0.0017)         \\
[1em]
Age 8-12  &  -0.0473\sym{***}&  -0.1694\sym{***}\\
          & (0.0015)         & (0.0018)         \\
[1em]
Age 13 or more&  -0.1084\sym{***}&  -0.2183\sym{***}\\
          & (0.0015)         & (0.0018)         \\
\hline
\(N\)     &   848871         &   966084         \\
m2        &  -2.5100         &  -4.1300         \\
Sargan test&  3.6e+03         &  2.0e+03         \\
df of Sargan Test& 495.0000         & 495.0000         \\
P value of Sargan test&   0.0000         &   0.0000         \\

   \bottomrule
  \end{tabular} 
\begin{tablenotes}
\small
\item Note: In this table, $m2$ is the Arellano-Bond test for zero autocorrelation in first-differenced errors for order two. Standard errors are in parentheses. ***, **, and * indicate statistically significant coefficients at 1\%, 5\%, and 10\% percent levels, respectively.
 \end{tablenotes}
% \end{threeparttable}
\end{table}

As $Emp_{e,t-1}$ is correlated with $\alpha_{i}$ because $Emp_{e,t-1}$ is a function of $\alpha_{i}$, \todo{This may need to be explained for non-economists - why is lagged employment of firm e a function of industry effect alpha i?}
OLS estimators are biased and inconsistent. 
To take this endogeneity bias into account, we use the estimation method from \textcite{RePEc:oup:restud:v:58:y:1991:i:2:p:277-297.} and find similar predictions (Table \ref{Dynamic - GMM}). To check the validity of the model, we use two tests. First, to test for autocorrelation, we use the test $m2$ by \textcite{RePEc:oup:restud:v:58:y:1991:i:2:p:277-297.}. In the table, we report the $z$ test statistic for $m2$ test for zero autocorrelation in the  first-differenced errors of order two. Second, we use the Sargan test to verify the validity of instrument subsets (shown in the last three rows in the table).

We furthermore estimate the model using the system GMM  method proposed by \textcite{RePEc:eee:econom:v:68:y:1995:i:1:p:29-51} and \textcite{RePEc:eee:econom:v:87:y:1998:i:1:p:115-143} and find similar predictions as before (Table \ref{Dynamic - system GMM}). 

\begin{table}[H]
  \centering
\begin{threeparttable}
 \caption{Regression coefficients (Dynamic - system GMM) for LEAP} \label{Dynamic - system GMM} \medskip
\renewcommand{\arraystretch}{1}
\begin{tabular}{l|c c| c c}
\toprule
\textbf{Independent Variables}&\multicolumn{2}{c|}{\textbf{LEAP}} &  \multicolumn{2}{c}{\textbf{CanSynLBD}}\\
\midrule
&\multicolumn{1}{c}{Private}&\multicolumn{1}{c}{Manufacturing}&\multicolumn{1}{c}{Private}&\multicolumn{1}{c}{Manufacturing}\\
\hline
AR(1) Coefficient&   0.0978\sym{***}&   0.1614\sym{***}&   0.5111\sym{***}&   0.5780\sym{***}\\
          & (0.0002)         & (0.0014)         & (0.0008)         & (0.0041)         \\
[1em]
Ln Pay    &   0.8854\sym{***}&   0.8161\sym{***}&   0.4562\sym{***}&   0.4022\sym{***}\\
          & (0.0002)         & (0.0012)         & (0.0006)         & (0.0033)         \\
[1em]
Age 3-4   &  -0.0555\sym{***}&  -0.1097\sym{***}&  -0.1828\sym{***}&  -0.3177\sym{***}\\
          & (0.0002)         & (0.0012)         & (0.0004)         & (0.0028)         \\
[1em]
Age 5-7   &  -0.0558\sym{***}&  -0.1201\sym{***}&  -0.1860\sym{***}&  -0.3408\sym{***}\\
          & (0.0002)         & (0.0013)         & (0.0005)         & (0.0031)         \\
[1em]
Age 8-12  &  -0.0548\sym{***}&  -0.1298\sym{***}&  -0.1875\sym{***}&  -0.3583\sym{***}\\
          & (0.0002)         & (0.0014)         & (0.0005)         & (0.0036)         \\
[1em]
Age 13 or more&  -0.0524\sym{***}&  -0.1317\sym{***}&  -0.1943\sym{***}&  -0.3747\sym{***}\\
          & (0.0002)         & (0.0016)         & (0.0006)         & (0.0041)         \\
\hline
\(N\)     & 15708195         &  1015293         & 13573225         &   959764         \\
m2        & -11.4300         &   1.3900         & -41.6000         &  -7.6700         \\
Sargan test&  7.7e+04         &  6.3e+03         &  1.8e+04         &  1.7e+03         \\
df of Sargan Test& 274.0000         & 274.0000         & 274.0000         & 274.0000         \\
P value of Sargan test&   0.0000         &   0.0000         &   0.0000         &   0.0000         \\

   \bottomrule
  \end{tabular} 
\begin{tablenotes}
\small
\item Note: An observation is an entity-year. In this table, $m2$ is the Arellano-Bond test for zero autocorrelation in first-differenced errors for order two. Standard errors are in parentheses. ***, **, and * indicate statistically significant coefficients at 1\%, 5\%, and 10\% percent levels, respectively.
 \end{tablenotes}
 \end{threeparttable}
\end{table}

\begin{table}[H]
  \centering
%\begin{threeparttable}
 \caption{Regression coefficients (Dynamic - system GMM) for GLBD} \label{Dynamic - system GMM} \medskip
\renewcommand{\arraystretch}{1}
\begin{tabular}{l|c| c}
\toprule
\textbf{Independent Variables}&\textbf{GLBD} &\textbf{GSynLBD}\\
\midrule
AR(1) Coefficient&   0.1883\sym{***}&   0.6140\sym{***}\\
          & (0.0021)         & (0.0027)         \\
[1em]
Ln Pay    &   0.6599\sym{***}&   0.3553\sym{***}\\
          & (0.0014)         & (0.0020)         \\
[1em]
Age 3-4   &  -0.0292\sym{***}&  -0.0934\sym{***}\\
          & (0.0011)         & (0.0013)         \\
[1em]
Age 5-7   &  -0.0512\sym{***}&  -0.1266\sym{***}\\
          & (0.0011)         & (0.0014)         \\
[1em]
Age 8-12  &  -0.0791\sym{***}&  -0.1545\sym{***}\\
          & (0.0011)         & (0.0015)         \\
[1em]
Age 13 or more&  -0.1400\sym{***}&  -0.2012\sym{***}\\
          & (0.0011)         & (0.0015)         \\
\hline
\(N\)     &   848871         &   966084         \\
m2        &  19.4900         &  -8.8300         \\
Sargan test&  4.5e+03         &  2.8e+03         \\
df of Sargan Test& 526.0000         & 526.0000         \\
P value of Sargan test&   0.0000         &   0.0000         \\

   \bottomrule
  \end{tabular} 
\begin{tablenotes}
\small
\item Note: An observation is an entity-year. In this table, $m2$ is the Arellano-Bond test for zero autocorrelation in first-differenced errors for order two. Standard errors are in parentheses. ***, **, and * indicate statistically significant coefficients at 1\%, 5\%, and 10\% percent levels, respectively.
 \end{tablenotes}
 %\end{threeparttable}
\end{table}

We also estimate above dynamic panel data model with a first-order moving average using appropriate instruments for both level and difference equation as proposed by \textcite{RePEc:eee:econom:v:68:y:1995:i:1:p:29-51} and \textcite{RePEc:eee:econom:v:87:y:1998:i:1:p:115-143}:

\todo{@ JA: $\alpha$ is used twice - not clear - please verify - I have added subscripts}
\begin{eqnarray}	
Emp_{et}&=&\alpha_i +\theta Emp_{e,t-1}+\lambda Pay_{et}+Age_{et}^{T}\beta+\lambda_t+\alpha_i+\epsilon_{et}+\gamma\epsilon_{e,t-1}
\end{eqnarray}

Table \ref{Dynamic - system GMM with MA(1)} shows that the CansynLBD provides similar predictions to the LEAP.

\begin{table}[H]
  \centering
\begin{threeparttable}
 \caption{Regression coefficients (Dynamic - system GMM with MA(1)) for LEAP} \label{Dynamic - system GMM with MA(1)} \medskip
\renewcommand{\arraystretch}{1}
\begin{tabular}{l|c c| c c}
\toprule
\textbf{Independent Variables}&\multicolumn{2}{c|}{\textbf{LEAP}} &  \multicolumn{2}{c}{\textbf{CanSynLBD}}\\
\midrule
&\multicolumn{1}{c}{Private}&\multicolumn{1}{c}{Manufacturing}&\multicolumn{1}{c}{Private}&\multicolumn{1}{c}{Manufacturing}\\
\hline
AR(1) Coefficient&   0.2005\sym{***}&   0.2821\sym{***}&   0.4850\sym{***}&   0.5737\sym{***}\\
          & (0.0007)         & (0.0040)         & (0.0012)         & (0.0059)         \\
[1em]
Ln Pay    &   0.8044\sym{***}&   0.7135\sym{***}&   0.4760\sym{***}&   0.4056\sym{***}\\
          & (0.0005)         & (0.0034)         & (0.0009)         & (0.0046)         \\
[1em]
Age 3-4   &  -0.1245\sym{***}&  -0.2033\sym{***}&  -0.1716\sym{***}&  -0.3158\sym{***}\\
          & (0.0005)         & (0.0032)         & (0.0006)         & (0.0037)         \\
[1em]
Age 5-7   &  -0.1328\sym{***}&  -0.2264\sym{***}&  -0.1733\sym{***}&  -0.3389\sym{***}\\
          & (0.0005)         & (0.0035)         & (0.0006)         & (0.0043)         \\
[1em]
Age 8-12  &  -0.1383\sym{***}&  -0.2454\sym{***}&  -0.1731\sym{***}&  -0.3560\sym{***}\\
          & (0.0006)         & (0.0039)         & (0.0007)         & (0.0051)         \\
[1em]
Age 13 or more&  -0.1441\sym{***}&  -0.2586\sym{***}&  -0.1774\sym{***}&  -0.3717\sym{***}\\
          & (0.0006)         & (0.0042)         & (0.0008)         & (0.0058)         \\
\hline
\(N\)     & 15708195         &  1015293         & 13573225         &   959764         \\
m2        &   8.2000         &   7.0600         & -40.0300         &  -6.6400         \\
Sargan test&  2.8e+04         &  2.3e+03         &  1.7e+04         &  1.3e+03         \\
df of Sargan Test& 251.0000         & 251.0000         & 251.0000         & 251.0000         \\
P value of Sargan test&   0.0000         &   0.0000         &   0.0000         &   0.0000         \\

   \bottomrule
  \end{tabular} 
\begin{tablenotes}
\small
\item Note: An observation is a firm and a year. In this table, $m2$ is the Arellano-Bond test for zero autocorrelation in first-differenced errors for order two. $LEAP$ is the Longitudinal Employment Analysis Program and $CanSynLBD$ is the Canadian synthetic database based on LEAP. In this table, we use 2015 vintage of LEAP and drop last year observation of each firm. Standard errors are in parentheses. ***, **, and * indicate statistically significant coefficients at 1\%, 5\%, and 10\% percent levels, respectively.
 \end{tablenotes}
 \end{threeparttable}
\end{table}

\begin{table}[H]
  \centering
%\begin{threeparttable}
 \caption{Regression coefficients (Dynamic - system GMM with MA(1)) for GLBD} \label{Dynamic - system GMM with MA(1)} \medskip
\renewcommand{\arraystretch}{1}
\begin{tabular}{l|c| c}
\toprule
\textbf{Independent Variables}&\textbf{GLBD} &\textbf{GSynLBD}\\
\midrule
AR(1) Coefficient&   0.3701\sym{***}&   0.5268\sym{***}\\
          & (0.0060)         & (0.0048)         \\
[1em]
Ln Pay    &   0.5349\sym{***}&   0.4202\sym{***}\\
          & (0.0041)         & (0.0036)         \\
[1em]
Age 3-4   &  -0.0594\sym{***}&  -0.0831\sym{***}\\
          & (0.0015)         & (0.0013)         \\
[1em]
Age 5-7   &  -0.0922\sym{***}&  -0.1105\sym{***}\\
          & (0.0018)         & (0.0015)         \\
[1em]
Age 8-12  &  -0.1252\sym{***}&  -0.1351\sym{***}\\
          & (0.0019)         & (0.0016)         \\
[1em]
Age 13 or more&  -0.1850\sym{***}&  -0.1802\sym{***}\\
          & (0.0019)         & (0.0017)         \\
\hline
\(N\)     &   848871         &   966084         \\
m2        &  19.0300         & -11.6900         \\
Sargan test&  3.1e+03         &  2.5e+03         \\
df of Sargan Test& 494.0000         & 494.0000         \\
P value of Sargan test&   0.0000         &   0.0000         \\

   \bottomrule
  \end{tabular} 
\begin{tablenotes}
\small
\item Note: An observation is a firm and a year. In this table, $m2$ is the Arellano-Bond test for zero autocorrelation in first-differenced errors for order two. Standard errors are in parentheses. ***, **, and * indicate statistically significant coefficients at 1\%, 5\%, and 10\% percent levels, respectively.
 \end{tablenotes}
% \end{threeparttable}
\end{table}