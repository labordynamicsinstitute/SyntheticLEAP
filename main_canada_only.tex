%!TeX TXS-program:bibliography = txs:///biber
\documentclass{article}
\usepackage[utf8]{inputenc}
\usepackage{booktabs, float, color, colortbl, graphicx, ragged2e, setspace, threeparttablex, threeparttable, longtable, placeins, booktabs, float, tabularx}
\usepackage[english]{babel}
\usepackage{adjustbox}
\usepackage{acronym}
\usepackage{csquotes}
\usepackage[plainpages=false]{hyperref}
\usepackage{todonotes}


\hypersetup{%
	%backref=true,%
    plainpages=false,%
	naturalnames=true,%
	bookmarksnumbered=true,%
	bookmarksopen=false,%
	plainpages=true,%
	colorlinks=true,%
	urlcolor=black,
	linkcolor=black,%
	filecolor=black,%
	citecolor=black,%
	%pagecolor=myblue,%
	%pdftitle={\mytitle},%
	pdfpagemode=UseOutlines%
	%pdfauthor={\myauthors},%
	%pdfsubject={\myshorttitle}
}
\usepackage[natbib,style=chicago-authordate,strict]{biblatex}

% bibliography
\addbibresource{paper.bib}

\acrodef{CDER}{Canadian Center for Data Development and Economic Research}
\acrodef{PEI}{Prince Edward Island}
\acrodef{LEAP}{Longitudinal Employment Analysis Program}

\newcommand{\sym}[1]{\rlap{#1}}
\title{Synthetic Data for Canadian Longitudinal Business Data }
\author{M. Jahangir Alam, Benoit Dostie, Lars Vilhuber}
\begin{document}
\maketitle
\setstretch{1.5}
\begin{abstract}
\noindent
Data on businesses collected by statistical agencies are challenging to protect. Many businesses have unique characteristics, distributions of employment, sales, and profits are highly skewed, and  most disclosure avoidance mechanisms  fail to strike an acceptable balance between usefulness and confidentiality protection. Often, only very few aggregate statistics are released, and access to confidential microdata can be burdensome. We document an experiment to create an analytically valid synthetic version of the Longitudinal Employment Analysis Program (LEAP). We implement the algorithm used by \citet{RePEc:bla:istatr:v:79:y:2011:i:3:p:362-384}, with minor adjustments. We document the analytical validity for various typical uses, as well as provide evidence of the  confidentiality of the database.


\end{abstract}
\newpage
\tableofcontents
\newpage
\section{Introduction}
There is growing demand for firm-level data allowing detailed studies of firm dynamics. Recent examples include \textcite{NBERc0480} who use cross-country firm-level data to study average post-entry behavior of young firms. \textcite{10.1257/aer.20141280} use the Business Dynamics Statistics (BDS) to show the role of firm size in firm dynamics. However, such studies are made difficult due to the limited or restricted access to firm-level data.

Data on businesses collected by statistical agencies are challenging to protect. Many businesses have unique characteristics, and distributions of employment, sales, and profits are highly skewed. Attackers wishing to conduct identification attacks often have access to much more information than for any individual. It is easy to find examples of firms and establishments that are so dominant in their industry or location that they would be immediately identified if  data were publicly released that included their survey responses or administratively collected data. Finally, there are also greater financial incentives to identifying the particulars of some firms and their competitors.

As a consequence, most disclosure avoidance mechanisms  fail to strike an acceptable balance between usefulness and confidentiality protection. Detailed aggregate statistics by geography or detailed industry classes  are rare, public-use microdata on business are virtually inexistant,\footnote{See \citet{NBERw22095} and \citet{startupcartography} for an example of scraped, public-use microdata.} and access to confidential microdata can be burdensome. It is not uncommon that access to establishment microdata, if granted at all, is provided through data enclaves (Research Data Centers), at headquarters of statistical agencies, or some other limited means, under strict security conditions. These restrictions on data access reduce the growth of knowledge by increasing the cost to researchers of accessing the data.

Synthetic microdata have been proposed as a secure mechanism to publish microdata \citep{drechsler2008,RePEc:taf:japsta:v:39:y:2012:i:2:p:243-265,NAP11844,SJIAOS-2014c}, based on suggestions and methods first proposed by \citet{rubin93} and \citet{little93}. Such data are  part of a broader discussion of how  to provide improved access to such datasets to researchers  \citep{Bender2009,Vilhuber2013,AbowdLane2004,AbowdSchmutte_BPEA2015}.\footnote{
	For a recent overview of some, see \citet{VilhuberAbowdReiter:Synthetic:SJIAOS:2016}. See \citet{dre:2011} for a review of the theory and applications of the synthetic data methodology.
	Other access methods include secure data enclaves (e.g., research data centers of the U.S. Federal Statistical System, of the  German Federal Employment Agency, others), and  remote submission system systems. We will comment on the latter in the conclusion. \todo{Make sure, we really do this in the conclusion or delete the last sentence.} }
For business data, synthetic business microdata were released in the United States \citep{KinneyEtAl2011} and in Germany  \citep{RePEc:iab:iabfme:201101_de} in 2011. The former dataset, called \ac{SynLBD}, was  released to an easily web-accessible computing environment \citep{AbowdVilhuber2010}, and combined with a validation mechanism.  By making disclosable synthetic microdata available through a remotely accessible data server, combined with a validation server, the SynLBD approach alleviates some of the access restrictions associated with economic data. The approach is mutually beneficial to both agency and researchers. Researchers can access public use servers at little or no cost, and can later validate their model-based inferences on the full confidential microdata. Details about the modeling strategies used for the \SynLBD can be found in  %
%\citet[henceforth KRRMJA]{KinneyEtAl2011} 
\citet{KinneyEtAl2011} 
and \citet{RePEc:cen:tnotes:11-01}.


In this article, we document an experiment to create analytically valid synthetic data, using the exact same model and methods previously used to create the \ac{SynLBD}, but applied to data from two different countries: Canada (\ac{LEAP}) and Germany (\ac{BHP}). We describe all three countries' data in Section~\ref{sec:data}. 


In Canada, the Canadian Center for Data Development and Economic Research (CDER) was created in 2011 to allow Statistics Canada to make better use of its business data holdings, without compromising security. Secure access  to business microdata for approved analytical research projects is done through a physical facility located in Statistics Canada’s headquarters. 

CDER implements many risks mitigation measures to alleviate the security risks specific to micro-level business data including limits on tabular outputs, centralized vetting, monitoring of programs logs. Access to the data is done through a Statistics Canada designed interface in which actual observations cannot be viewed. But the most significant barrier to access remains the  cost of traveling to Ottawa.

The Institute for Employment Research (IAB) in Germany also strictly regulates the access to its business data. All business data can  be accessed exclusively onsite at the research data center (RDC) and only after the research proposal has been approved by the Federal Ministry of Labour and Social Affairs. All output is carefully checked by staff at the RDC and only cleared output can be removed  from the RDC. 

The experiment aims not so much at finding the \textit{best} synthetic data method for each file, but rather to assess the effectiveness of using a `pre-packaged' method to cost-effectively generate synthetic data. In particular, while we could have used newer implementations of methods combined with a pre-defined or automated model \citep{JSSv074i11,Raab_Nowok_Dibben_2018}, we chose to use the exact SAS code used to create the original \ac{SynLBD}. A brief synopsis of the  method, and any adjustments we made to take into account structural data differences, are described in Section~\ref{sec:methodology}.


%\todo{do modify according to what was exactly done in the respective cases of Canada and Germany}

We verify the analytical validity of the synthetic data files so created along a variety of measures. First, we show how well average firm characteristics (gross employment, total payroll) in the synthetic data  match those from the original data. We also consider how well the synthetic data replicates various measures of firm dynamics (entry and exit rates) and job flows (job creation and destruction rate). Second, we compute an aggregate measure of distributional fit between the synthetic and confidential data ($pMSE$). Finally, we estimates for models of economic growth, and assess how these estimates vary between both data sets using dynamic panel data models. 

To assess how protective the newly created synthetic database is, we estimate the probability that the synthetic first year equals the true first year given the synthetic fist year.

% and find that those probabilities are quite low except for the first year included in the respective databases. 
%The probability for the first year is higher because of censoring and lack of previous information.

The rest of the paper is organized as follows. Section 2 describes the different data sources and summarizes which steps were taken to harmonize the datasets prior to the actual synthesis. Section 3 provides some background on the synthesis methods, limitations in the applications, and a discussion of some of the measures, which are used in Section 4 to measure the analytical validity of the generated datasets. Preliminary results regarding the achieved level of protection are included in Section 5. The paper concludes with...
\todo{Write something about conclusion}


\section{Data Description: the Canadian LEAP}

In this section, we briefly describe the structure of the three data sources before summarizing the steps taken to harmonize the different data sources and discussing the limitations in the current application.

\subsection{United States: \acf{LBD}}

The \ac{LBD} \citep{LBD} is created from the U.S. Census Bureau's \ac{BR} by creating longitudinal links of establishments using name and address matching. The database has information on birth, death, location, industry,  firm  affiliation of employer establishments, and ownership by multi-establishment firms, as well as their employment over time, for nearly all sectors of the economy from 1976 through 2015 (as of this writing). It serves as a key linkage file as well as a research dataset in its own right for numerous research articles, as well as a tabulation input to the U.S. Census Bureau's \acl{BDS} \citep[\acs{BDS}]{BDS}. Other statistics created from the underlying Business Register include the \acl{CBP} \citep[\acs{CBP}]{CBP} and the \acl{SBUSB} \citep[\acs{SBUSB}]{SBUSB}. For a full description, readers should consult  \citet{RePEc:cen:wpaper:02-17,}. The key variables of interest for this experiment are birth and death dates, payroll, employment, and the industry coding of the establishment.  \citet{SJIAOS-2014d} explore a possible expansion of the synthesis methods described later to include location and firm affiliation. Note that information on payroll and employment does not come from individual-level wage records, as is the case for both the Canadian and German datasets described below, as well as for the \acl{QWI} \citep{AbowdEtAl2009} derived from the \acl{LEHD} \citep[LEHD]{RePEc:cen:wpaper:18-27} in the United States. \todo{Lars, can you include a sentence where the data come from, then?} Thus, methods that connect establishments based on labor flows \citep{BenedettoEtAl2007,RePEc:iab:iabfme:201006_en} are not employed. We also note that payroll is the cumulative sum of wages paid over the entire calendar year, whereas employment is measured as of March 12 of each year.


%The US LBD was created in the early 2000s (Miranda and Jarmin, 2002), following previous research files with more restrictive coverage. At its core, it is a research database containing longitudinally linked data records from a statistical business register of establishments. Breaks of longitudinal links are resolved using probabilistic name and address matching. The variables currently in the LBD are industry, annual payroll, employment, geography, birth year, death year, and firm structure. Though it has very few variables on the database itself, it serves as a backbone for many linkages into establishment and firm surveys and censuses such as business dynamics and job flows. 

%The fundamental structure of the LBD (and thus the SynLBD) is a longitudinal file on economic entities, where each entity has a start and end date and a small number of key attributes that evolve over time. Hypothetically, this structure is shared by many other longitudinal panels, such as panels of jobs or of residences. We should note that it does not apply to data structures like a linked employer-employee database, since there are no linkages between entities at a point in time. Thus, using concepts from graph theory, it is a mapping of a network that contains only nodes, and no edges. These structural characteristics are relevant for any attempt to generalize the synthesizing methodology to other contexts, such as matched employer-employee data (but see Barrientos et al., 2017).

%The primary goal of the SynLBD project is to create partially synthetic microdata on establishments for public release, allowing researchers easier access for the implementation of a wide (unconstrained) range of models with analytically valid inferences about the underlying population, while protecting against re-identification of any given unit or its attributes. There are multiple reasons why a public release of such data is desirable. The US LBD is one of most requested datasets in the Federal Statistical Research Data Centers (FSRDCs), but access through FSRDC the is still subject to long approval processes. In many European countries, access to data on business registers is arduous or impossible for researchers. Access through commercial providers is possible (Bureau Van Dijk), but coverage is generally poor.




\subsection{Canada: \acf{LEAP}}


The \ac{LEAP} \citep{StatisticsCanada2019} contains information on annual employment for each employer business in all sectors of the Canadian economy. It covers incorporated and unincorporated businesses that issue at least one annual statement of remuneration paid (T4 slips) in any given calendar year. It excludes self-employed individuals or partnerships with non-salaried participants.

To construct the \ac{LEAP}, Statistics Canada uses three sources of information: (1) T4 administrative data  from the Canada Revenue Agency (CRA), (2) data from Statistics Canada's \acl{BR} \citep{StatisticsCanada2019a}, and (3) data from  Statistics Canada's \acf{SEPH} \citep{StatisticsCanada2019b}. 
%
%\begin{description}
%\item[T4] 
In general, all employers in Canada provide employees with a T4 slip if they paid employment income, taxable allowances and benefits, or any other remuneration in any calendar year. The T4 information is reported to the tax agency, which in turn provides this information to Statistics Canada. 
%
%\item[BR] 
The Business Register is Statistics Canada's central repository of baseline information on businesses and institutions operating in Canada. It is used as the survey frame for all business related data sets.
%
%\item[SEPH] 
The objective of the \ac{SEPH} is to provide monthly information on the level of earnings, the number of jobs, and hours worked by detailed industry at the national and provincial levels. To do so, it combines a census of approximately one million payroll deductions provided by the CRA, and the Business Payrolls Survey, a sample of 15,000 establishments.  
%\end{description}

The core \ac{LEAP}  contains four variables (1) a longitudinal Business Register Identifier (LBRID), (2) an industry classification, (3) payroll and (4) a measure of employment. 
%
%\begin{enumerate}
%\item 
The LBRID uniquely identifies each enterprise and is derived from the Business Register. To avoid ``false'' deaths and births due to mergers, restructuring or changes in reporting practices, Statistics Canada uses employment flows. Similar to \citet{BenedettoEtAl2007} and \citet{RePEc:iab:iabfme:201006_en}, the method  compares the cluster of workers in each newly identified enterprise with all the clusters of workers in firms from the previous year. This comparison yields a new identifier (LBRID) derived from those of the \ac{BR}.
%\item 
The industry classification comes from the \ac{BR} for single-industry firms. If a firm operates in multiple industries, information on payroll from the \ac{SEPH} is used to identify the industry in which the firm pays the highest payroll. Prior to 1991, information on industry was based on the SIC,  but it is currently based on the  North American Industrial Classification System (NAICS). We use the information at the NAICS four-digit (industry group) level. 
%\item 
The firm's payroll is measured as the sum of all T4s  reported to the CRA for the calendar year.
%\item 
Employment is measured either using \ac{ILU} or \ac{ALU}. \acp{ALU} are obtained by dividing the payroll by the average annual earnings in its industry/province/class category computed using the \ac{SEPH}. \acp{ILU} are a head count of the number of T4 issued by the enterprise, with employees working for multiple employers split proportionately across firms according to their total annual payroll earned in each firm. 


%\end{enumerate}
% Commented out: too much an advertisement... jars with the rest.
% The \ac{LEAP} is the only data set in Canada  that allows research on a variety of themes, like employment growth, industry turnover, firm survival, job creation and job destruction, etc. 

For the purpose of this experiment,  we exclude the public sector (NAICS 61, 62, and 91), even though it is contained in the database, because it may not be accurately captured \citep{StatisticsCanada2019}. Statistics Canada does not publish any statistics for those sectors.

\subsection{Germany: \acf{BHP}}

Because the Institute for Employment Research is affiliated with the German Ministry of Labour and Social Affairs, it collects very little data at the business or establishment level.\footnote{Exceptions are the IAB Establishment Panel \citep{IABEstabPanel} and the IAB Job Vacancy Survey \citep{JVS}.}

The core database for the Establishment History Panel is the German Social Security Data  (GSSD), which is based on the integrated notification procedure for the health, pension and unemployment insurances,   introduced in  1973. Employers report information on all their employees, by establishment. Aggregating this information via an establishment identifier yields the Establishment History Panel \citep[German abbreviation: BHP]{BHP}. We used data from  1975 until 2008, which at the time this project started was the most current data available for research. Information for the former Eastern German States is limited to the years 1992-2008. 

Due to the purpose and structure of the GSSD, some variables present in the \ac{LBD} are not available on the  \ac{BHP}. Firm-level information is not captured, and it is thus not known whether establishments are part of a multi-establishment employer. In 1999, reporting requirements were extended to all establishments; prior to that date, only establishments that  had at least one employee covered by social security on the reference date June 30 of each year were subject to filing requirements. Payroll and employment are both based on a reference date of June 30, and are thus consistent point-in-time measures. 
\citet{SJIAOS-2014b} describe adjustments made to the \ac{BHP} for this project, including estimating full-year payroll, creating time-consistent industry identifiers, and applying employment flow methods \citep{RePEc:iab:iabfme:201006_en} to adjust for spurious births and deaths in establishment identifiers. 
Industries are identified according to the NAICS classification system on the five digit level. However, we aggregated the industry information for this project by only using the first four digits of the coding system.

\todo{Still need to add the info somewhere that the synthesis models are run independently within each four digit WZ and that we only use data from two industries in our evaluations.}





\subsection{Overview}

After the various standardizations and choices made above, the data structure is intended to be comparable, as summarized in Table~\ref{tab:common_Variable}.

% Table tab:LEAP_Variable
%\begin{table}[H]
  \centering\footnotesize
  \caption{CanSynLEAP variable descriptions}  \label{tab:LEAP_Variable} \medskip
  \renewcommand{\arraystretch}{1}
  \begin{tabular}{l  c c c c c}
    \toprule
    \textbf{Name}&\textbf{Type}&\textbf{Description}&\textbf{Notation}&\textbf{Action}\\
    \midrule
synid&Identifier&Unique random number for enterprise&&Created\\
NAICS&Categorical&4 digit industry code&$x_{1}$&Unmodified\\
Firstyear&Categorical&First year enterprise is observed &$y_{1}$&Synthesized\\
Lastyear&Categorical&Last year enterprise is observed &$y_{2}$&Synthesized\\
Year&Categorical&Year dating of annual variables&&Created\\
ALU&Continuous&Average Labor Unit (annual)&$y_{3}$&Synthesized\\
Payroll&Continuous&Payroll (annual)&$x_{4}$&Synthesized\\
   \bottomrule
  \end{tabular} 
\\
Note: Variables denoted with $y_{i}$ are synthesized and variables denoted with $x_{i}$ are not synthesized. 
\end{table}
\begin{table}[H]
  \centering\footnotesize
  \caption{Variable descriptions and comparison}  \label{tab:common_Variable} \medskip
  \renewcommand{\arraystretch}{1}
 \setlength{\tabcolsep}{4.5pt}
 \begin{tabular}{l  c c c c c c c}
    \toprule
    \textbf{Name}&\textbf{Type} &\textbf{Description} &\textbf{US} & \textbf{Canada} &\textbf{Germany} &\textbf{Nature}\\
    \midrule
Entity Identifier& identifier& & Establishment & Employer & Establishment &Created\\
Industry code&Categorical& Various across countries &SIC3 & NAICS4 & WZ2003 &Unmodified\\
             &           &                          &(3-digit )& (4-digit) &(4-digit)  &\\
First year&Categorical&First year entity is observed &\multicolumn{3}{c}{--- firstyear ---}&Synthesized\\
Last year&Categorical&Last year entity is observed &\multicolumn{3}{c}{--- lastyear ---}&Synthesized\\
Year&Categorical&Year dating of annual variables&\multicolumn{3}{c}{--- year ---}&Derived\\
Employment & Continuous & Employment measure & Count & ALU* & Count & Synthesized \\
            &            &                    & (March 15) &(annual)& (June 30)&\\
Payroll&Continuous&  Payroll (annual)& Reported & Computed & Computed,  &Synthesized\\
       &          &                  &          &          & Adjusted\\
   \bottomrule
  \end{tabular} 
  \flushleft{* ALU = Average Labour Unit. See text for additional explanations.}
%\\
%Note: Variables denoted with $y_{i}$ are synthesized and variables denoted with $x_{i}$ are not synthesized. 
\end{table}


\section{Data Description: the German Case}
%
In this section, we briefly describe the structure of the three data sources before summarizing the steps taken to harmonize the different data sources and discussing the limitations in the current application.

\subsection{United States: \acf{LBD}}

The \ac{LBD} \citep{LBD} is created from the U.S. Census Bureau's \ac{BR} by creating longitudinal links of establishments using name and address matching. The database has information on birth, death, location, industry,  firm  affiliation of employer establishments, and ownership by multi-establishment firms, as well as their employment over time, for nearly all sectors of the economy from 1976 through 2015 (as of this writing). It serves as a key linkage file as well as a research dataset in its own right for numerous research articles, as well as a tabulation input to the U.S. Census Bureau's \acl{BDS} \citep[\acs{BDS}]{BDS}. Other statistics created from the underlying Business Register include the \acl{CBP} \citep[\acs{CBP}]{CBP} and the \acl{SBUSB} \citep[\acs{SBUSB}]{SBUSB}. For a full description, readers should consult  \citet{RePEc:cen:wpaper:02-17,}. The key variables of interest for this experiment are birth and death dates, payroll, employment, and the industry coding of the establishment.  \citet{SJIAOS-2014d} explore a possible expansion of the synthesis methods described later to include location and firm affiliation. Note that information on payroll and employment does not come from individual-level wage records, as is the case for both the Canadian and German datasets described below, as well as for the \acl{QWI} \citep{AbowdEtAl2009} derived from the \acl{LEHD} \citep[LEHD]{RePEc:cen:wpaper:18-27} in the United States. \todo{Lars, can you include a sentence where the data come from, then?} Thus, methods that connect establishments based on labor flows \citep{BenedettoEtAl2007,RePEc:iab:iabfme:201006_en} are not employed. We also note that payroll is the cumulative sum of wages paid over the entire calendar year, whereas employment is measured as of March 12 of each year.


%The US LBD was created in the early 2000s (Miranda and Jarmin, 2002), following previous research files with more restrictive coverage. At its core, it is a research database containing longitudinally linked data records from a statistical business register of establishments. Breaks of longitudinal links are resolved using probabilistic name and address matching. The variables currently in the LBD are industry, annual payroll, employment, geography, birth year, death year, and firm structure. Though it has very few variables on the database itself, it serves as a backbone for many linkages into establishment and firm surveys and censuses such as business dynamics and job flows. 

%The fundamental structure of the LBD (and thus the SynLBD) is a longitudinal file on economic entities, where each entity has a start and end date and a small number of key attributes that evolve over time. Hypothetically, this structure is shared by many other longitudinal panels, such as panels of jobs or of residences. We should note that it does not apply to data structures like a linked employer-employee database, since there are no linkages between entities at a point in time. Thus, using concepts from graph theory, it is a mapping of a network that contains only nodes, and no edges. These structural characteristics are relevant for any attempt to generalize the synthesizing methodology to other contexts, such as matched employer-employee data (but see Barrientos et al., 2017).

%The primary goal of the SynLBD project is to create partially synthetic microdata on establishments for public release, allowing researchers easier access for the implementation of a wide (unconstrained) range of models with analytically valid inferences about the underlying population, while protecting against re-identification of any given unit or its attributes. There are multiple reasons why a public release of such data is desirable. The US LBD is one of most requested datasets in the Federal Statistical Research Data Centers (FSRDCs), but access through FSRDC the is still subject to long approval processes. In many European countries, access to data on business registers is arduous or impossible for researchers. Access through commercial providers is possible (Bureau Van Dijk), but coverage is generally poor.




\subsection{Canada: \acf{LEAP}}


The \ac{LEAP} \citep{StatisticsCanada2019} contains information on annual employment for each employer business in all sectors of the Canadian economy. It covers incorporated and unincorporated businesses that issue at least one annual statement of remuneration paid (T4 slips) in any given calendar year. It excludes self-employed individuals or partnerships with non-salaried participants.

To construct the \ac{LEAP}, Statistics Canada uses three sources of information: (1) T4 administrative data  from the Canada Revenue Agency (CRA), (2) data from Statistics Canada's \acl{BR} \citep{StatisticsCanada2019a}, and (3) data from  Statistics Canada's \acf{SEPH} \citep{StatisticsCanada2019b}. 
%
%\begin{description}
%\item[T4] 
In general, all employers in Canada provide employees with a T4 slip if they paid employment income, taxable allowances and benefits, or any other remuneration in any calendar year. The T4 information is reported to the tax agency, which in turn provides this information to Statistics Canada. 
%
%\item[BR] 
The Business Register is Statistics Canada's central repository of baseline information on businesses and institutions operating in Canada. It is used as the survey frame for all business related data sets.
%
%\item[SEPH] 
The objective of the \ac{SEPH} is to provide monthly information on the level of earnings, the number of jobs, and hours worked by detailed industry at the national and provincial levels. To do so, it combines a census of approximately one million payroll deductions provided by the CRA, and the Business Payrolls Survey, a sample of 15,000 establishments.  
%\end{description}

The core \ac{LEAP}  contains four variables (1) a longitudinal Business Register Identifier (LBRID), (2) an industry classification, (3) payroll and (4) a measure of employment. 
%
%\begin{enumerate}
%\item 
The LBRID uniquely identifies each enterprise and is derived from the Business Register. To avoid ``false'' deaths and births due to mergers, restructuring or changes in reporting practices, Statistics Canada uses employment flows. Similar to \citet{BenedettoEtAl2007} and \citet{RePEc:iab:iabfme:201006_en}, the method  compares the cluster of workers in each newly identified enterprise with all the clusters of workers in firms from the previous year. This comparison yields a new identifier (LBRID) derived from those of the \ac{BR}.
%\item 
The industry classification comes from the \ac{BR} for single-industry firms. If a firm operates in multiple industries, information on payroll from the \ac{SEPH} is used to identify the industry in which the firm pays the highest payroll. Prior to 1991, information on industry was based on the SIC,  but it is currently based on the  North American Industrial Classification System (NAICS). We use the information at the NAICS four-digit (industry group) level. 
%\item 
The firm's payroll is measured as the sum of all T4s  reported to the CRA for the calendar year.
%\item 
Employment is measured either using \ac{ILU} or \ac{ALU}. \acp{ALU} are obtained by dividing the payroll by the average annual earnings in its industry/province/class category computed using the \ac{SEPH}. \acp{ILU} are a head count of the number of T4 issued by the enterprise, with employees working for multiple employers split proportionately across firms according to their total annual payroll earned in each firm. 


%\end{enumerate}
% Commented out: too much an advertisement... jars with the rest.
% The \ac{LEAP} is the only data set in Canada  that allows research on a variety of themes, like employment growth, industry turnover, firm survival, job creation and job destruction, etc. 

For the purpose of this experiment,  we exclude the public sector (NAICS 61, 62, and 91), even though it is contained in the database, because it may not be accurately captured \citep{StatisticsCanada2019}. Statistics Canada does not publish any statistics for those sectors.

\subsection{Germany: \acf{BHP}}

Because the Institute for Employment Research is affiliated with the German Ministry of Labour and Social Affairs, it collects very little data at the business or establishment level.\footnote{Exceptions are the IAB Establishment Panel \citep{IABEstabPanel} and the IAB Job Vacancy Survey \citep{JVS}.}

The core database for the Establishment History Panel is the German Social Security Data  (GSSD), which is based on the integrated notification procedure for the health, pension and unemployment insurances,   introduced in  1973. Employers report information on all their employees, by establishment. Aggregating this information via an establishment identifier yields the Establishment History Panel \citep[German abbreviation: BHP]{BHP}. We used data from  1975 until 2008, which at the time this project started was the most current data available for research. Information for the former Eastern German States is limited to the years 1992-2008. 

Due to the purpose and structure of the GSSD, some variables present in the \ac{LBD} are not available on the  \ac{BHP}. Firm-level information is not captured, and it is thus not known whether establishments are part of a multi-establishment employer. In 1999, reporting requirements were extended to all establishments; prior to that date, only establishments that  had at least one employee covered by social security on the reference date June 30 of each year were subject to filing requirements. Payroll and employment are both based on a reference date of June 30, and are thus consistent point-in-time measures. 
\citet{SJIAOS-2014b} describe adjustments made to the \ac{BHP} for this project, including estimating full-year payroll, creating time-consistent industry identifiers, and applying employment flow methods \citep{RePEc:iab:iabfme:201006_en} to adjust for spurious births and deaths in establishment identifiers. 
Industries are identified according to the NAICS classification system on the five digit level. However, we aggregated the industry information for this project by only using the first four digits of the coding system.

\todo{Still need to add the info somewhere that the synthesis models are run independently within each four digit WZ and that we only use data from two industries in our evaluations.}





\subsection{Overview}

After the various standardizations and choices made above, the data structure is intended to be comparable, as summarized in Table~\ref{tab:common_Variable}.

% Table tab:LEAP_Variable
%\begin{table}[H]
  \centering\footnotesize
  \caption{CanSynLEAP variable descriptions}  \label{tab:LEAP_Variable} \medskip
  \renewcommand{\arraystretch}{1}
  \begin{tabular}{l  c c c c c}
    \toprule
    \textbf{Name}&\textbf{Type}&\textbf{Description}&\textbf{Notation}&\textbf{Action}\\
    \midrule
synid&Identifier&Unique random number for enterprise&&Created\\
NAICS&Categorical&4 digit industry code&$x_{1}$&Unmodified\\
Firstyear&Categorical&First year enterprise is observed &$y_{1}$&Synthesized\\
Lastyear&Categorical&Last year enterprise is observed &$y_{2}$&Synthesized\\
Year&Categorical&Year dating of annual variables&&Created\\
ALU&Continuous&Average Labor Unit (annual)&$y_{3}$&Synthesized\\
Payroll&Continuous&Payroll (annual)&$x_{4}$&Synthesized\\
   \bottomrule
  \end{tabular} 
\\
Note: Variables denoted with $y_{i}$ are synthesized and variables denoted with $x_{i}$ are not synthesized. 
\end{table}
\begin{table}[H]
  \centering\footnotesize
  \caption{Variable descriptions and comparison}  \label{tab:common_Variable} \medskip
  \renewcommand{\arraystretch}{1}
 \setlength{\tabcolsep}{4.5pt}
 \begin{tabular}{l  c c c c c c c}
    \toprule
    \textbf{Name}&\textbf{Type} &\textbf{Description} &\textbf{US} & \textbf{Canada} &\textbf{Germany} &\textbf{Nature}\\
    \midrule
Entity Identifier& identifier& & Establishment & Employer & Establishment &Created\\
Industry code&Categorical& Various across countries &SIC3 & NAICS4 & WZ2003 &Unmodified\\
             &           &                          &(3-digit )& (4-digit) &(4-digit)  &\\
First year&Categorical&First year entity is observed &\multicolumn{3}{c}{--- firstyear ---}&Synthesized\\
Last year&Categorical&Last year entity is observed &\multicolumn{3}{c}{--- lastyear ---}&Synthesized\\
Year&Categorical&Year dating of annual variables&\multicolumn{3}{c}{--- year ---}&Derived\\
Employment & Continuous & Employment measure & Count & ALU* & Count & Synthesized \\
            &            &                    & (March 15) &(annual)& (June 30)&\\
Payroll&Continuous&  Payroll (annual)& Reported & Computed & Computed,  &Synthesized\\
       &          &                  &          &          & Adjusted\\
   \bottomrule
  \end{tabular} 
  \flushleft{* ALU = Average Labour Unit. See text for additional explanations.}
%\\
%Note: Variables denoted with $y_{i}$ are synthesized and variables denoted with $x_{i}$ are not synthesized. 
\end{table}


\section{Methodology}
%The US SynLBD was released in 2010 to the Cornell SDS. The Census Bureau’s Disclosure Review Board (DRB), as well as the Internal Revenue Service (IRS), classified SynLBD as public-use, but access is controlled due to concerns about the quality of the data. There are no disclosure concerns but researchers are cautioned not to trust results as if they were created by a traditional public-use file without going through the validation process. For similar reasons, the preparation of tabular data based on the synthetic data is strongly discouraged, and are not validated. Nevertheless, the synthetic data are of much easier access than the confidential data.


The Synthetic LBD is derived from the LBD as a partially synthetic database with analytic validity, by synthesizing the life-span of establishments, as well as the evolution of their employment, conditional on industry. Geography is not synthesized, but is suppressed from the released file \citep{RePEc:cen:tnotes:11-01}. The current version, is based on the Standard Industrial Classification (SIC) and extends through 2000. \citet{RePEc:cen:wpaper:14-12} describes efforts to create a new version of the Synthetic LBD, using a longer time  series (through 2010) and newer industry coding (NAICS), while also adjusting and extending the models for  improved  analytic validity and  the imputation of additional variables. In this paper, we refer to and re-use the older methodology, which we will call \SynLBD. Our emphasis is on the comparability of results obtained for a given methodology across the various applications.
  

\deleted{We can currently distinguish between two methods to create synthetic data.}\todo{Since we do not describe the two approaches, I would suggest dropping this sentence.} The general approach to data synthesis is to generate a joint posterior predictive distribution of $Y|X$ where $Y$ are variables to be synthesized and $X$ are unsynthesized variables. The synthetic data are generated by sampling new values from this distribution. In \SynLBD, variables are synthesized in a sequential fashion, with categorical variables being generally processed first using a variant of Dirichlet-Multinomial models. Continuous variables are then synthesized using a normal linear regression model with kernel density-based transformation \citep{WOODCOCK20094228}.\footnote{\textcite{RePEc:cen:wpaper:14-12} shift  to a Classification and Regression Trees (CART) model with Bayesian bootstrap. } \SynLBD{} is implemented in SAS\texttrademark, which is frequently used in national statistical offices.

To evaluate whether synthetic data algorithms developed in the U.S. can be adapted to generate similar synthetic data for other countries, \textcite{RePEc:cen:wpaper:14-13} implement \SynLBD{} to the German Longitudinal Business Database (GLBD). In this paper, we extend the analysis from the earlier paper, and extend the application to the Canadian context (SynLEAP). 


\subsection{Harmonizing and Preprocessing}

In all countries, the underlying data provides annual measures. However, \SynLBD{} assumes a longitudinal (wide) structure of the dataset, with invariant industry (and location). In all cases, the modal industry is chosen to represent the entity's industrial activity. 

Further adjustments made to the \ac{BHP} for this project, include estimating full-year payroll, creating time-consistent geographic information, and applying employment flow methods \citep{RePEc:iab:iabfme:201006_en} to adjust for spurious births and deaths in establishment identifiers. \citet{SJIAOS-2014b} provide a detailed description of the steps taken to harmonize the input data. 

\subsection{Limitations}

In both cases, we encountered various technical and data-driven limitations. In all countries, the first year and last year is occasionally problematic, and were dropped. Furthermore, in all countries, the synthesis of certain industries failed to complete. In both Canada and the US, this number is less than 10. In Canada, they account for about 7 percent of the total number of observations (see Table \ref{tab:Synthesized_observations} in the Appendix).

In the German case, our experiments were limited to only a handful of industries, due to a combination of time and software availability factors. The results should still be considered preliminary. In both countries, as outlined in Section~\ref{sec:data}, there are subtle but potentially important differences in the various variable definitions. Industry coding differs across all three countries, and the level of detail in each of the industry codings may affect the success and precision of the synthesis.\footnote{\textcite{StatisticsCanada1991}, when comparing the 1987 US \ac{SIC} to the 1980 Canadian \ac{SIC},  already pointed out that the degree of specialization, the organization of production, and the size of the respective markets differed. Thus, the density of establishments within each of the chosen categories is likely to affect the quality of the synthesis.} 

Furthermore, due to the nature of the underlying data, entities are establishments in Germany and the US, but employers in Canada. \SynLBD{} should work on any level of entity aggregation (see \citet{RePEc:cen:wpaper:14-12} for an application to hierarchical firm data with both firm/employer and establishment level imputation). However, it may yet again affect the observed density of the data within industry-year categories, and therefore the overall comparability. 

Both the German and the Canadian data experience some level of industry coding change, which may affect the classification of some entities. 

Finally, due to a feature of \SynLBD{} that we do not fully understand, the last year of the data generally was of poor quality. For some industry-country pairs, this also happened in the first year. We dropped those observations. 

\subsection{Measuring outcomes}

In order to assess the outcomes of the experiment, we inspect analytical validity by various measures, the extent of confidentiality protection, and the utility for model development. To check analytical validity, we compare basic univariate time series between the synthetic and confidential data (employment, entity entry and exit rates, job creation and destruction rates), and the distribution of entities (firms and establishment, depending on country),  employment, and payroll across time by industry. 

To provide a more comprehensive measure of  quality of the synthetic data relative to the confidential data, we compute the $pMSE$ \parencite[propensity score mean-squared error,][]{Woo_Reiter_Oganian_Karr_2009,SnokeSlavkovic2018,Snoke_RSSA2018}: the mean-squared error of the predicted probabilities (i.e., propensity scores) for those two databases. Specifically, $pMSE$ is a metric to assess how well we are able to discern the high distributional similarity between synthetic data and confidential data. 

We follow  \textcite{SnokeSlavkovic2018} to calculate the $pMSE$. This method involved the following steps: 
\todo{This notation clashes with the earlier one. We need to consolidate notation. I've replaced "it" with "et" (entity) reserving the "i" for industry}
\begin{enumerate}
    \item Append the $n_1$ rows of the confidential database $X$ to the $n_2$ rows of the synthetic database $X^s$ to create $X^{comb}$ with $N=n_1 + n_2$ rows.
    \item Create a variable $I_{et}$ denoting membership of an observation for entity $e$ in the component databases,  $I_{et}=\{1: X^{comb}_{et} \in X^s\}$. $I_{et}$ takes on values of $1$ for the synthetic database and $0$ for the confidential database. 
    \item Fit the following model to predict $I$
    \begin{eqnarray}	
        I_{et} & = &\alpha + \beta_{1} Emp_{et} + \beta_{2} Pay_{et} + Age_{et}^{T}\beta_{3} + \lambda_t + \alpha_i + \epsilon_{et} \label{pMSE}
     \end{eqnarray}
         where $Emp_{et}$ is  log employment  of entity $e$ in year $t$, $Pay_{et}$ is  log payroll of entity $e$ in year $t$, $Age_{et}$ is a vector of age classes of entity $e$ in year $t$, $\lambda_t$ is a year fixed effect, $\alpha_i$ is an unobserved time-invariant industry-specific effect, and $\epsilon_{et}$ is the disturbance term of entity $e$ in year $t$. 
    \item Calculate the predicted probabilities, $\hat{I}_{it}$
    \item Compute the $pMSE=\frac{1}{N}\sum_{i=1}^N(\hat{p}_i - n_1/N)^2$
\end{enumerate}
If $n_1 = n_2$, $pMSE$ = 0 means every $\hat{p}_i = 0.5$, and the two databases are distributionally indistinguishable and we can conclude that the analytical validity is high, as the variables included in the propensity model cannot predict the data source.


For a more complex assessment, we compute a dynamic panel data model of economic (employment) growth on each dataset. We then assess fit of the model in two ways. \todo{What is the other assessment you had in mind?} First, analytic validity --- statistical precision --- can be assessed using confidence interval overlap measures   \citep{tas2006}. \todo{Do we compute them? If not, we should drop this. If we want we could include a short discussion why CIO is not a helpful measure here}
We compute the \emph{interval overlap measure} $J_{k,m}$ for parameter $k$ in model $m$. Consider the overlap of confidence intervals $(L,U)$ for $\beta_{k,m}$ (estimated from the confidential data) and $(L^{*},U^{*})$ for $\beta_{k,m}^*$ (from the synthetic data). Let $L^{over} = \max (L,L^{*} )$ and $U^{over} = \min (U,U^{*})$. Then the average overlap in confidence intervals is
$$
J_{k,m}^{*} = \frac{1}{2} \left [ \frac{U^{over} - L^{over}}{U-L} + \frac{U^{over} - L^{over}}{U^*-L ^*}        \right ]
$$
We can also compute an overall score by  averaging $J_{k,m}^{*}$ over all  parameters. 

\section{Analytical validity}

\newcommand{\TableNote}{$LEAP$ is the Longitudinal Employment Analysis Program and $CanSynLBD$ is the Canadian synthetic database based on LEAP. Here, we use the 2015 vintage of LEAP and drop the last year observations.}

\subsection{Firm Characteristics}

The CanSynLBD and LEAP generally provide comparable inferences on aggregate means and correlations. For example, Figures \ref{GrossEmploymentPrivate} and \ref{GrossEmploymentManufacturing} show that gross employment levels for each year in the CanSynLBD are very close to those in the LEAP. However, the manufacturing sector shows closer patterns than the private sector.\footnote{The private sector comprises all industries including the manufacturing sector except the public sector  (NAICS 61, 62, and 91)} We find similar results for total payroll (Figures \ref{TotalPayrollPrivate} and  \ref{TotalPayrollManufacturing}) .

\todo{Why is manufacturing always below, but overall employment crosses? Which industries are driving that?} \todo{JA: I checked before, but I could not able to identify any specific reason.}
\begin{figure} [H]
\centering
\caption{Gross employment level by year (private)} \label{GrossEmploymentPrivate}
\includegraphics[height=2.8in, width=.7\linewidth]{graphs/Gross_employment_level_by_year_private_bw.pdf} 
\begin{minipage}{0.85\textwidth}
{\footnotesize Note: \TableNote \par}
\end{minipage}
\end{figure}



\begin{figure} [H]
\centering
\caption{Gross employment level by year (manufacturing)} \label{GrossEmploymentManufacturing}
\includegraphics[height=2.8in, width=.7\linewidth]{graphs/Gross_employment_level_by_year_manufacturing_bw.pdf} 
\begin{minipage}{0.85\textwidth}
{\footnotesize Note: \TableNote  \par}
\end{minipage}
\end{figure}


\begin{figure} [H]
\centering
\caption{Total payroll by year (private)} \label{TotalPayrollPrivate}
\includegraphics[height=2.8in, width=.7\linewidth]{graphs/Total_payroll_by_year_private_bw.pdf} 
\begin{minipage}{0.85\textwidth}
{\footnotesize Note: \TableNote \par}
\end{minipage}
\end{figure}
\begin{figure} [H]
\centering
\caption{Total payroll by year (manufacturing)} \label{TotalPayrollManufacturing}
\includegraphics[height=2.8in, width=.7\linewidth]{graphs/Total_payroll_by_year_manufacturing_bw.pdf} 
\begin{minipage}{0.85\textwidth}
{\footnotesize Note: \TableNote \par}
\end{minipage}
\end{figure}

Figures \ref{FirmSharePrivate} and \ref{FirmShareManufacturing} plot the share of firms by two-digit industry and year for both the CanSynLBD and the LEAP databaseand show that those shares clustering along the 45-degree line.

\begin{figure} [H]
\centering
\caption{Share of firms by NAICS two-digit and year (private)} \label{FirmSharePrivate}
\includegraphics[height=2.8in, width=.7\linewidth]{graphs/Share_of_firms_by_NAICS_two-digit_and_year_private_bw.pdf} 
\begin{minipage}{0.85\textwidth}
{\footnotesize Note: \TableNote \par}
\end{minipage}
\end{figure}


\vspace{-15.5pt}
\begin{figure} [H]
\centering
\caption{Share of firms by NAICS two-digit and year (manufacturing)} \label{FirmShareManufacturing}
\includegraphics[height=2.8in, width=.7\linewidth]{graphs/Share_of_firms_by_NAICS_two-digit_and_year_Manufacturing_bw.pdf} 
\begin{minipage}{0.85\textwidth}
{\footnotesize Note: \TableNote \par}
\end{minipage}
\end{figure}

Figures \ref{EmploymentSharePrivate} and \ref{EmploymentShareManufacturing} plot the share of employment by two-digit industry and year for both the CanSynLBD and the LEAP database
\footnote{We define the share of employment as $x_{its} = X_{its}/\sum_{i} \sum_{t} X_{its}$, where $i$ are two-digit NAICS industries, $t$ are the years in-sample, $s$ indicates whether it is in the synthetic or confidential data, and $X_{its}$ is the total employment for industry $i$ and year $t$ for either the synthetic or confidential data $s$.} and show that those shares do not cluster along the 45-degree line. However, this hides significant differences between sectors as, for the share of employment for the manufacturing sector, we do observe more clustering along the 45-degrees.
\begin{figure} [H]
\centering
\caption{Share of employment by NAICS two-digit and year (private)} \label{EmploymentSharePrivate}
\includegraphics[height=2.8in, width=.7\linewidth]{graphs/Share_of_employment_by_NAICS_two-digit_and_year_private_bw.pdf} 
\begin{minipage}{0.85\textwidth}
{\footnotesize Note: \TableNote \par}
\end{minipage}
\end{figure}
\vspace{-15.5pt}
\begin{figure} [H]
\centering
\caption{Share of employment by NAICS two-digit and year (manufacturing)} \label{EmploymentShareManufacturing}
\includegraphics[height=2.8in, width=.7\linewidth]{graphs/Share_of_employment_by_NAICS_two-digit_and_year_Manufacturing_bw.pdf} 
\begin{minipage}{0.85\textwidth}
{\footnotesize Note: \TableNote \par}
\end{minipage}
\end{figure}

Figures \ref{PayrollSharePrivate} and \ref{PayrollShareManufacturing} plot the share of payroll by two-digit industry and year for both CanSynLBD and LEAP database and show that those shares do not cluster along the 45-degree line. Again, we do notice that for the share of employment for the manufacturing sector, we do observe more clustering along the 45-degrees.
\begin{figure} [H]
\centering
\caption{Share of payroll by NAICS two-digit and year (private)} \label{PayrollSharePrivate}
\includegraphics[height=2.8in, width=.7\linewidth]{graphs/Share_of_payroll_by_NAICS_two-digit_and_year_private_bw.pdf} 
\begin{minipage}{0.85\textwidth}
{\footnotesize Note: \TableNote \par}
\end{minipage}
\end{figure}
\vspace{-15.5pt}
\begin{figure} [H]
\centering
\caption{Share of payroll by NAICS two-digit and year (manufacturing)} \label{PayrollShareManufacturing}
\includegraphics[height=2.8in, width=.7\linewidth]{graphs/Share_of_payroll_by_NAICS_two-digit_and_year_Manufacturing_bw.pdf} 
\begin{minipage}{0.85\textwidth}
{\footnotesize Note: \TableNote \par}
\end{minipage}
\end{figure}

\subsection{Firm Dynamics}
To assess how well the CanSynLBD captures firm dynamics, we also compute entry and exit rates of the private sector by year. Table \ref{FirmDynamics} shows that those rates for CanSynLBD are similar to LEAP database. To show further those rates are similar, we compute the divergence of entry rate as the entry rate of CanSynLBD net the entry rate of LEAP as well as the divergence of exit rate as the exit rate of CanSynLBD net the exit rate of LEAP (see Figure \ref{Divergence}).

\begin{table}[H]
  \centering
\begin{threeparttable}
 \caption{Entry and exit rates by year} \label{FirmDynamics} \medskip
\renewcommand{\arraystretch}{1}
\begin{tabular}{l|c c| c c| c c}
\toprule
&\multicolumn{2}{c|}{\textbf{LEAP}} &  \multicolumn{2}{c|}{\textbf{CanSynLBD}}&  \multicolumn{2}{c}{\textbf{Divergence}}\\
\textbf{Year}&\textbf{Entry Rate}&\textbf{Exit Rate}&\textbf{Entry Rate}&\textbf{Exit Rate} &\textbf{Entry Rate}&\textbf{Exit Rate}\\
\midrule
1992&11.77&11.72&11.16&11.71&-0.60&-0.00\\
1993&11.81&11.61&10.84&12.18&-0.97&0.57\\
1994&12.04&11.79&11.57&12.01&-0.47&0.22\\
1995&11.94&12.09&11.69&12.26&-0.25&0.17\\
1996&12.91&10.31&12.62&10.64&-0.29&0.32\\
1997&13.18&9.75&13.03&10.21&-0.15&0.47\\
1998&12.48&10.89&12.97&10.13&0.50&-0.75\\
1999&12.00&10.66&12.16&9.97&0.16&-0.69\\
2000&11.80&10.51&11.59&9.70&-0.20&-0.82\\
2001&11.44&10.20&11.33&9.52&-0.12&-0.68\\
2002&11.39&9.91&11.10&9.03&-0.29&-0.89\\
2003&11.17&10.21&10.52&9.37&-0.65&-0.84\\
2004&12.13&9.76&10.94&9.57&-1.20&-0.20\\
2005&11.92&10.07&11.07&9.86&-0.84&-0.21\\
2006&11.81&9.96&11.15&9.34&-0.66&-0.62\\
2007&12.28&9.80&10.99&9.31&-1.29&-0.49\\
2008&11.60&10.14&10.78&9.75&-0.82&-0.40\\
2009&10.77&9.93&9.99&9.81&-0.78&-0.12\\
2010&10.80&9.75&9.91&9.65&-0.89&-0.10\\
2011&10.62&9.79&9.73&10.00&-0.89&0.21\\
2012&10.60&9.76&10.02&10.20&-0.58&0.44\\
2013&10.16&9.71&9.95&10.32&-0.21&0.62\\
2014&9.93&10.11&9.26&10.70&-0.67&0.59\\

   \bottomrule
  \end{tabular} 
\begin{tablenotes}
\small
\item Note: \TableNote  We calculate the divergence of entry rate as the entry rate of CanSynLBD net the entry rate of LEAP and the divergence of exit rate as the exit rate of CanSynLBD net the exit rate of LEAP.
 \end{tablenotes}
 \end{threeparttable}
\end{table}

\begin{figure} [H]
\centering
\caption{Divergence of exit and entry rate between LEAP and CanSynLBD} \label{Divergence}
\includegraphics[height=2.8in, width=.7\linewidth]{graphs/Divergence_of_exit_and_entry_rate_between_LEAP_and_CanSynLBD_bw.pdf} 
\begin{minipage}{0.85\textwidth}
{\footnotesize Note: \TableNote  We calculate the divergence of entry rate as the entry rate of CanSynLBD net the entry rate of LEAP and the divergence of exit rate as the exit rate of CanSynLBD net the exit rate of LEAP. \par}
\end{minipage}
\end{figure}

\subsection{Dynamics of Job Flows}

One of the most important applications of LEAP is to generate statistics that describe job flows. Following \cite{DavisHaltiwangerSchuh}, the job creation is defined as the sum of all employment gains from expanding firms from year $t-1$ to year $t$ including entry firms. The job destruction rate is defined as the sum of all employment losses from contracting firms from year $t-1$ to year $t$ including exiting firms. Net job creation is the job creation rate minus the job destruction rate. Figures \ref{JobCreationPrivate} and \ref{JobCreationManufacturing} show the job creation rates from the CanSynLBD compared againg those of the LEAP. These figures show that the manufacturing sector has closer pattern than the private sector. We find a similar patterns for net job creation rates (Figures \ref{NetJobCreationPrivate} and  \ref{NetJobCreationManufacturing}).

\begin{figure} [H]
\centering
\caption{Job creation rate by year (private)} \label{JobCreationPrivate}
\includegraphics[height=2.8in, width=.7\linewidth]{graphs/Job_creation_rate_by_year_private_bw.pdf} 
\begin{minipage}{0.85\textwidth}
{\footnotesize Note: \TableNote \par}
\end{minipage}
\end{figure}

\begin{figure} [H]
\centering
\caption{Job creation rate  by year (manufacturing)} \label{JobCreationManufacturing}
\includegraphics[height=2.8in, width=.7\linewidth]{graphs/Job_creation_rate_by_year_Manufacturing_bw.pdf} 
\begin{minipage}{0.85\textwidth}
{\footnotesize Note: \TableNote \par}
\end{minipage}
\end{figure}

\todo{LV regraph, dropping last year} \todo{JA: Should we mention this in the text including reasons if we drop the last year here?}
\begin{figure} [H]
\centering
\caption{Net job creation rate by year (private)} \label{NetJobCreationPrivate}
\includegraphics[height=2.8in, width=.7\linewidth]{graphs/Net_job_creation_rate_by_year_private_bw.pdf} 
\begin{minipage}{0.85\textwidth}
{\footnotesize Note: \TableNote \par}
\end{minipage}
\end{figure}
\begin{figure} [H]
\centering
\caption{Net job creation rate  by year (manufacturing)} \label{NetJobCreationManufacturing}
\includegraphics[height=2.8in, width=.7\linewidth]{graphs/Net_job_creation_rate_by_year_Manufacturing_bw.pdf} 
\begin{minipage}{0.85\textwidth}
{\footnotesize Note: $LEAP$ is the Longitudinal Employment Analysis Program and $CanSynLBD$ is the Canadian synthetic database based on LEAP. In this graph, we use 2015 vintage of LEAP for the manufacturing sector and drop last year observation of each firm. \par}
\end{minipage}
\end{figure}

\subsection{pMSE}

To compare the quality of the synthetic data relative to the confidential data, we compute $pMSE$, which is the mean-squared error of the predicted probabilities (i.e., propensity scores) for those two databases. Specifically, $pMSE$ is a metric to assess how well we are able to discern the high distributional similarity between synthetic data and confidential data. 

We follow the method by \textcite{SnokeSlavkovic2018} to calculate the $pMSE$. This method involved the following steps: 
\begin{enumerate}
    \item Append the $n_1$ rows of the confidential database $X$ to the $n_2$ rows of the synthetic database $X^s$ to create $X^{comb}$ with $N=n_1 + n_2$ rows.
    \item Create an indicator variable, $I$, to $X^{comb}$ subject to $I=\{1: X^{comb} \in X^s\}$. This means that we create an indicator variable of $1$ for the synthetic database and $0$ for the confidential database. 
    \item Fit the following model to predict $I$
    \begin{eqnarray}	
        I & = &\alpha + ALU_{it} + \lambda Pay_{it} + Age_{it}^{T}\beta + \lambda_t + \alpha_s + \epsilon_{it} \label{pMSE}
     \end{eqnarray}
     \todo{BD: I don't understand why the dependant variable has no indices?}
     \todo{JA: This is an indicator variable of $1$ for the synthetic database and $0$ for the confidential database. In this case, we need to add one more index in all variables as well.}
    where $ALU_{it}$ is the logarithm of average labour unit (ALU) of firm $i$ in year $t$, $Pay_{it}$ is the logarithm of payroll of firm $i$ in year $t$, $Age_{it}$ is a vector of dummy variables for age of firm $i$ in year $t$, $\lambda_t$ is the year fixed effect, $\alpha_s$ is an unobserved time-invariant industry-specific effect, and $\epsilon_{it}$ is the disturbance term of firm $i$ in year $t$. 
    \item calculate the predicted probabilities, $\hat{p}_i$ for each row of $X^{comb}$
    \item Compute the $pMSE=\frac{1}{N}\sum_{i=1}^N(\hat{p}_i - 0.5)^2$
\end{enumerate}
A $pMSE$ = 0 means every $\hat{p}_i = 0.5$. 

To compute the $pMSE$, we estimate equation \ref{pMSE} using both the logit and probit models. Table \ref{pMSE_regression} \todo{BD: Are those coefficients or marginal effects? Does it make sense to show coefficients? Or are we interested only in the last row? If we are interested in the coefficient, why is there no discussion of those?} \todo{Those are coefficients. I think we are interested in the last row.} shows the calculated value of $pMSE$, which is lower for the manufacturing sector than the public sector in both regressions. This is because, as we explained before, the synthetic data mirrors the original data more closely in the case of the manufacturing sector.

\begin{table}[H]
  \centering
\begin{threeparttable}
 \caption{pMSE estimates} \label{pMSE_regression} \medskip
\renewcommand{\arraystretch}{1}
\begin{tabular}{l|c c| c c}
\toprule
\textbf{Independent Variables}&\multicolumn{2}{c|}{\textbf{Logistic Regression}} &  \multicolumn{2}{c}{\textbf{Probit Regression}}\\
\midrule
          &\multicolumn{1}{c}{Manufacturing}&\multicolumn{1}{c}{Private}&\multicolumn{1}{c}{Manufacturing}&\multicolumn{1}{c}{Private}\\
\hline
Ln ALU    &   0.1580\sym{***}&   0.7138\sym{***}&   0.1003\sym{***}&   0.4390\sym{***}\\
          & (0.0039)         & (0.0010)         & (0.0024)         & (0.0006)         \\
[1em]
Ln Pay    &   0.0039         &  -0.4426\sym{***}&   0.0012         &  -0.2691\sym{***}\\
          & (0.0037)         & (0.0010)         & (0.0023)         & (0.0006)         \\
[1em]
Age 3-4   &   0.0392\sym{***}&   0.0972\sym{***}&   0.0252\sym{***}&   0.0618\sym{***}\\
          & (0.0078)         & (0.0017)         & (0.0049)         & (0.0010)         \\
[1em]
Age 5-7   &  -0.0382\sym{***}&   0.0477\sym{***}&  -0.0233\sym{***}&   0.0309\sym{***}\\
          & (0.0073)         & (0.0016)         & (0.0045)         & (0.0010)         \\
[1em]
Age 8-12  &  -0.1258\sym{***}&  -0.0263\sym{***}&  -0.0781\sym{***}&  -0.0152\sym{***}\\
          & (0.0071)         & (0.0015)         & (0.0044)         & (0.0009)         \\
[1em]
Age 13 or more&  -0.2190\sym{***}&  -0.1024\sym{***}&  -0.1365\sym{***}&  -0.0627\sym{***}\\
          & (0.0074)         & (0.0016)         & (0.0046)         & (0.0010)         \\
\hline
\(N\)     &  2243011         & 34638723         &  2243011         & 34638723         \\
pseudo \(R^{2}\)&   0.0112         &   0.0318         &   0.0112         &   0.0320         \\
pMSE      &   0.0041         &   0.0121         &   0.0041         &   0.0124         \\

   \bottomrule
  \end{tabular} 
\begin{tablenotes}
\small
\item Note: An observation is a firm and a year of both synthetic and original databases. In all specifications, we include both time and industry fixed effects. Standard errors are in parentheses. In this table, we use 2015 vintage of LEAP to create the synthetic database and drop last year observation of each firm. ***, **, and * indicate statistically significant coefficients at 1\%, 5\%, and 10\% percent levels, respectively.
 \end{tablenotes}
 \end{threeparttable}
\end{table}

\subsection{Regression Analysis}

To assess how well the CanSynLBD captures variability in economic growth due to industry and firm age, we estimate the following dynamic panel data model:
\begin{eqnarray}	
ALU_{it} & = & \alpha + \theta ALU_{i,t-1} + \lambda Pay_{it} + Age_{it}^{T}\beta + \lambda_t + \alpha_s + \epsilon_{it}
\end{eqnarray}
where $ALU_{it}$ is the logarithm of average labour unit (ALU) of firm $i$ in year $t$, $ALU_{i,t-1}$ is the logarithm of last year's average labour unit (ALU) of firm $i$, $Pay_{it}$ is the logarithm of payroll of firm $i$ in year $t$, $Age_{it}$ is a vector of dummy variables for age of firm $i$ in year $t$, $\lambda_t$ is the year fixed effect, $\alpha_s$ is an unobserved time-invariant industry-specific effect, and $\epsilon_{it}$ is the disturbance term of firm $i$ in year $t$. 

\begin{table}[H]
  \centering
\begin{threeparttable}
 \caption{Regression coefficients (OLS)} \label{OLS} \medskip
\renewcommand{\arraystretch}{1}
\begin{tabular}{l|c c| c c}
\toprule
\textbf{Independent Variables}&\multicolumn{2}{c|}{\textbf{LEAP}} &  \multicolumn{2}{c}{\textbf{CanSynLBD}}\\
\midrule
&\multicolumn{1}{c}{Private}&\multicolumn{1}{c}{Manufacturing}&\multicolumn{1}{c}{Private}&\multicolumn{1}{c}{Manufacturing}\\
\hline
AR(1) Coefficient&   0.2031\sym{***}&   0.2481\sym{***}&   0.3970\sym{***}&   0.4405\sym{***}\\
          & (0.0001)         & (0.0005)         & (0.0002)         & (0.0007)         \\
[1em]
Ln Pay    &   0.7847\sym{***}&   0.7300\sym{***}&   0.5481\sym{***}&   0.5228\sym{***}\\
          & (0.0001)         & (0.0005)         & (0.0002)         & (0.0006)         \\
[1em]
Age 3-4   &  -0.1202\sym{***}&  -0.1717\sym{***}&  -0.1223\sym{***}&  -0.2340\sym{***}\\
          & (0.0003)         & (0.0014)         & (0.0004)         & (0.0016)         \\
[1em]
Age 5-7   &  -0.1260\sym{***}&  -0.1891\sym{***}&  -0.1235\sym{***}&  -0.2507\sym{***}\\
          & (0.0003)         & (0.0014)         & (0.0004)         & (0.0016)         \\
[1em]
Age 8-12  &  -0.1268\sym{***}&  -0.1973\sym{***}&  -0.1169\sym{***}&  -0.2551\sym{***}\\
          & (0.0003)         & (0.0013)         & (0.0004)         & (0.0016)         \\
[1em]
Age 13 or more&  -0.1246\sym{***}&  -0.1992\sym{***}&  -0.1101\sym{***}&  -0.2577\sym{***}\\
          & (0.0003)         & (0.0014)         & (0.0004)         & (0.0017)         \\
\hline
\(N\)     & 15708195         &  1015293         & 13573225         &   959764         \\
\(R^{2}\) &   0.9696         &   0.9743         &   0.9444         &   0.9523         \\

   \bottomrule
  \end{tabular} 
\begin{tablenotes}
\small
\item Note: In all specifications, we include both year and industry fixed effects. Standard errors are in parentheses. $LEAP$ is the Longitudinal Employment Analysis Program and $CanSynLBD$ is the Canadian synthetic database based on LEAP. In this table, we use the 2015 vintage of LEAP and drop last year observation of each firm. ***, **, and * indicate statistically significant coefficients at 1\%, 5\%, and 10\% percent levels, respectively.
 \end{tablenotes}
 \end{threeparttable}
\end{table}

We estimate the model separately on LEAP and CanSynLBD data for the private and manufacturing sectors and find that the CansynLBD data provides similar predictions to LEAP data (Tables  \ref{OLS}).\todo{compute overlap interval} \todo{JA: @Lars, I think you mentioned once that you would like to calculate this. I could calculate using the method explained in the appendix.}

\begin{table}[H]
  \centering
\begin{threeparttable}
 \caption{Regression coefficients (Dynamic)} \label{Dynamic - GMM} \medskip
\renewcommand{\arraystretch}{1}
\begin{tabular}{l|c c| c c}
\toprule
\textbf{Independent Variables}&\multicolumn{2}{c|}{\textbf{LEAP}} &  \multicolumn{2}{c}{\textbf{CanSynLBD}}\\
\midrule
&\multicolumn{1}{c}{Private}&\multicolumn{1}{c}{Manufacturing}&\multicolumn{1}{c}{Private}&\multicolumn{1}{c}{Manufacturing}\\
\hline
AR(1) Coefficient&   0.0805\sym{***}&   0.1189\sym{***}&   0.5722\sym{***}&   0.5425\sym{***}\\
          & (0.0003)         & (0.0018)         & (0.0024)         & (0.0084)         \\
[1em]
Ln Pay    &   0.8991\sym{***}&   0.8523\sym{***}&   0.4101\sym{***}&   0.4302\sym{***}\\
          & (0.0002)         & (0.0015)         & (0.0018)         & (0.0067)         \\
[1em]
Age 3-4   &  -0.0450\sym{***}&  -0.0797\sym{***}&  -0.2075\sym{***}&  -0.2972\sym{***}\\
          & (0.0002)         & (0.0014)         & (0.0010)         & (0.0051)         \\
[1em]
Age 5-7   &  -0.0438\sym{***}&  -0.0860\sym{***}&  -0.2129\sym{***}&  -0.3162\sym{***}\\
          & (0.0002)         & (0.0015)         & (0.0011)         & (0.0059)         \\
[1em]
Age 8-12  &  -0.0418\sym{***}&  -0.0923\sym{***}&  -0.2187\sym{***}&  -0.3294\sym{***}\\
          & (0.0003)         & (0.0017)         & (0.0013)         & (0.0070)         \\
[1em]
Age 13 or more&  -0.0379\sym{***}&  -0.0898\sym{***}&  -0.2318\sym{***}&  -0.3414\sym{***}\\
          & (0.0003)         & (0.0019)         & (0.0015)         & (0.0080)         \\
\hline
\(N\)     & 15708195         &  1015293         & 13573225         &   959764         \\
m2        & -14.5000         &  -2.2200         & -27.5400         &  -9.4400         \\
Sargan test&  6.9e+04         &  4.6e+03         &  1.5e+04         &  1.5e+03         \\
df of Sargan Test& 252.0000         & 252.0000         & 252.0000         & 252.0000         \\
P value of Sargan test&   0.0000         &   0.0000         &   0.0000         &   0.0000         \\

   \bottomrule
  \end{tabular} 
\begin{tablenotes}
\small
\item Note: In this table, $m2$ is the Arellano-Bond test for zero autocorrelation in first-differenced errors for order two. $LEAP$ is the Longitudinal Employment Analysis Program and $CanSynLBD$ is the Canadian synthetic database based on LEAP. In this graph, we use the 2015 vintage of LEAP and drop last year observation of each firm. Standard errors are in parentheses. ***, **, and * indicate statistically significant coefficients at 1\%, 5\%, and 10\% percent levels, respectively.
 \end{tablenotes}
 \end{threeparttable}
\end{table}

As $ALU_{st-1}$ is correlated with $\alpha_{s}$ because $ALU_{st-1}$ is a function of $\alpha_{s}$, OLS estimators are biased and inconsistent. 
To take this endogeneity bias into account, we use the estimation method from \textcite{RePEc:oup:restud:v:58:y:1991:i:2:p:277-297.} and find similar predictions (Table \ref{Dynamic - GMM}). To check the validity of the model, we use two tests. First, to test for autocorrelation, we use the test $m2$ by \textcite{RePEc:oup:restud:v:58:y:1991:i:2:p:277-297.}. In the table, we report the $z$ test statistic for $m2$ test for zero autocorrelation in the  first-differenced errors of order two. Second, we use the Sargan test to verify the validity of instrument subsets (showned in the last three rows in the table).

We furthermore estimate the model using the system GMM  method proposed by \textcite{RePEc:eee:econom:v:68:y:1995:i:1:p:29-51} and \textcite{RePEc:eee:econom:v:87:y:1998:i:1:p:115-143} and find similar predictions as before (Table \ref{Dynamic - system GMM}). 

\begin{table}[H]
  \centering
\begin{threeparttable}
 \caption{Regression coefficients (Dynamic - system GMM)} \label{Dynamic - system GMM} \medskip
\renewcommand{\arraystretch}{1}
\begin{tabular}{l|c c| c c}
\toprule
\textbf{Independent Variables}&\multicolumn{2}{c|}{\textbf{LEAP}} &  \multicolumn{2}{c}{\textbf{CanSynLBD}}\\
\midrule
&\multicolumn{1}{c}{Private}&\multicolumn{1}{c}{Manufacturing}&\multicolumn{1}{c}{Private}&\multicolumn{1}{c}{Manufacturing}\\
\hline
AR(1) Coefficient&   0.0978\sym{***}&   0.1614\sym{***}&   0.5111\sym{***}&   0.5780\sym{***}\\
          & (0.0002)         & (0.0014)         & (0.0008)         & (0.0041)         \\
[1em]
Ln Pay    &   0.8854\sym{***}&   0.8161\sym{***}&   0.4562\sym{***}&   0.4022\sym{***}\\
          & (0.0002)         & (0.0012)         & (0.0006)         & (0.0033)         \\
[1em]
Age 3-4   &  -0.0555\sym{***}&  -0.1097\sym{***}&  -0.1828\sym{***}&  -0.3177\sym{***}\\
          & (0.0002)         & (0.0012)         & (0.0004)         & (0.0028)         \\
[1em]
Age 5-7   &  -0.0558\sym{***}&  -0.1201\sym{***}&  -0.1860\sym{***}&  -0.3408\sym{***}\\
          & (0.0002)         & (0.0013)         & (0.0005)         & (0.0031)         \\
[1em]
Age 8-12  &  -0.0548\sym{***}&  -0.1298\sym{***}&  -0.1875\sym{***}&  -0.3583\sym{***}\\
          & (0.0002)         & (0.0014)         & (0.0005)         & (0.0036)         \\
[1em]
Age 13 or more&  -0.0524\sym{***}&  -0.1317\sym{***}&  -0.1943\sym{***}&  -0.3747\sym{***}\\
          & (0.0002)         & (0.0016)         & (0.0006)         & (0.0041)         \\
\hline
\(N\)     & 15708195         &  1015293         & 13573225         &   959764         \\
m2        & -11.4300         &   1.3900         & -41.6000         &  -7.6700         \\
Sargan test&  7.7e+04         &  6.3e+03         &  1.8e+04         &  1.7e+03         \\
df of Sargan Test& 274.0000         & 274.0000         & 274.0000         & 274.0000         \\
P value of Sargan test&   0.0000         &   0.0000         &   0.0000         &   0.0000         \\

   \bottomrule
  \end{tabular} 
\begin{tablenotes}
\small
\item Note: An observation is a firm and a year. In this table, $m2$ is the Arellano-Bond test for zero autocorrelation in first-differenced errors for order two. $LEAP$ is the Longitudinal Employment Analysis Program and $CanSynLBD$ is the Canadian synthetic database based on LEAP. In this table, we use 2015 vintage of LEAP and drop last year observation of each firm. Standard errors are in parentheses. ***, **, and * indicate statistically significant coefficients at 1\%, 5\%, and 10\% percent levels, respectively.
 \end{tablenotes}
 \end{threeparttable}
\end{table}

We also estimate above dynamic panel data model with a first-order moving average using appropriate instruments for both level and difference equation as proposed by \textcite{RePEc:eee:econom:v:68:y:1995:i:1:p:29-51} and \textcite{RePEc:eee:econom:v:87:y:1998:i:1:p:115-143}:
\begin{eqnarray}	
ALU_{it}&=&\alpha +\theta ALU_{i,t-1}+\lambda Pay_{it}+Age_{it}^{T}\beta+\lambda_t+\alpha_s+\epsilon_{it}+\gamma\epsilon_{it-1}
\end{eqnarray}

Table \ref{Dynamic - system GMM with MA(1)} shows that the CansynLBD provides similar predictions to the LEAP.

\begin{table}[H]
  \centering
\begin{threeparttable}
 \caption{Regression coefficients (Dynamic - system GMM with MA(1))} \label{Dynamic - system GMM with MA(1)} \medskip
\renewcommand{\arraystretch}{1}
\begin{tabular}{l|c c| c c}
\toprule
\textbf{Independent Variables}&\multicolumn{2}{c|}{\textbf{LEAP}} &  \multicolumn{2}{c}{\textbf{CanSynLBD}}\\
\midrule
&\multicolumn{1}{c}{Private}&\multicolumn{1}{c}{Manufacturing}&\multicolumn{1}{c}{Private}&\multicolumn{1}{c}{Manufacturing}\\
\hline
AR(1) Coefficient&   0.2005\sym{***}&   0.2821\sym{***}&   0.4850\sym{***}&   0.5737\sym{***}\\
          & (0.0007)         & (0.0040)         & (0.0012)         & (0.0059)         \\
[1em]
Ln Pay    &   0.8044\sym{***}&   0.7135\sym{***}&   0.4760\sym{***}&   0.4056\sym{***}\\
          & (0.0005)         & (0.0034)         & (0.0009)         & (0.0046)         \\
[1em]
Age 3-4   &  -0.1245\sym{***}&  -0.2033\sym{***}&  -0.1716\sym{***}&  -0.3158\sym{***}\\
          & (0.0005)         & (0.0032)         & (0.0006)         & (0.0037)         \\
[1em]
Age 5-7   &  -0.1328\sym{***}&  -0.2264\sym{***}&  -0.1733\sym{***}&  -0.3389\sym{***}\\
          & (0.0005)         & (0.0035)         & (0.0006)         & (0.0043)         \\
[1em]
Age 8-12  &  -0.1383\sym{***}&  -0.2454\sym{***}&  -0.1731\sym{***}&  -0.3560\sym{***}\\
          & (0.0006)         & (0.0039)         & (0.0007)         & (0.0051)         \\
[1em]
Age 13 or more&  -0.1441\sym{***}&  -0.2586\sym{***}&  -0.1774\sym{***}&  -0.3717\sym{***}\\
          & (0.0006)         & (0.0042)         & (0.0008)         & (0.0058)         \\
\hline
\(N\)     & 15708195         &  1015293         & 13573225         &   959764         \\
m2        &   8.2000         &   7.0600         & -40.0300         &  -6.6400         \\
Sargan test&  2.8e+04         &  2.3e+03         &  1.7e+04         &  1.3e+03         \\
df of Sargan Test& 251.0000         & 251.0000         & 251.0000         & 251.0000         \\
P value of Sargan test&   0.0000         &   0.0000         &   0.0000         &   0.0000         \\

   \bottomrule
  \end{tabular} 
\begin{tablenotes}
\small
\item Note: An observation is a firm and a year. In this table, $m2$ is the Arellano-Bond test for zero autocorrelation in first-differenced errors for order two. $LEAP$ is the Longitudinal Employment Analysis Program and $CanSynLBD$ is the Canadian synthetic database based on LEAP. In this table, we use 2015 vintage of LEAP and drop last year observation of each firm. Standard errors are in parentheses. ***, **, and * indicate statistically significant coefficients at 1\%, 5\%, and 10\% percent levels, respectively.
 \end{tablenotes}
 \end{threeparttable}
\end{table}

\subsection{Confidentiality protection}

In this section, we estimate the probability that the synthetic first year equals the true first year, given the synthetic first year.\todo{BD: That is some strange wording} \todo{JA: I got this definition from the U.S. SynLBD paper.} Tables \ref{ProbabilityPrivate} and \ref{ProbabilityManufacturing} show that these probabilities are quite low except for the first year.\todo{BD: Are we worried about this?} \todo{JA: Somebody asked me when I presented at StatCan.} The probability for the first year is higher because of censoring and lack of previous information.

\begin{table}[H]
\centering\footnotesize
\caption{Observed firm births given synthetic births (private)} \label{ProbabilityPrivate} \medskip
\renewcommand{\arraystretch}{1}
\begin{tabular}{c c| c c c}
\toprule
\multicolumn{2}{c|}{\textbf{First (Birth) Year}} &  \multicolumn{3}{c}{\textbf{\% of Births over NAICS}}\\
\textbf{Synthetic}&\textbf{Actual}&\textbf{Minimum}&\textbf{Mean}&\textbf{Maximum}\\
\midrule
1991&1991&0.00&27.69&83.02\\
1992&1992&0.00&3.37&11.11\\
1993&1993&0.00&3.79&33.33\\
1994&1994&0.00&3.73&33.33\\
1995&1995&0.00&3.86&20.00\\
1996&1996&0.00&4.25&33.33\\
1997&1997&0.00&4.10&16.94\\
1998&1998&0.00&4.41&25.00\\
1999&1999&0.00&4.23&33.33\\
2000&2000&0.00&3.41&25.00\\
2001&2001&0.00&2.73&22.22\\
2002&2002&0.00&2.65&25.00\\
2003&2003&0.00&2.22&10.00\\
2004&2004&0.00&2.60&17.86\\
2005&2005&0.00&2.71&20.00\\
2006&2006&0.00&2.83&50.00\\
2007&2007&0.00&2.90&33.33\\
2008&2008&0.00&2.38&20.00\\
2009&2009&0.00&2.47&50.00\\
2010&2010&0.00&2.12&33.33\\
2011&2011&0.00&2.65&50.00\\
2012&2012&0.00&2.41&20.00\\
2013&2013&0.00&2.48&25.00\\
2014&2014&0.00&2.23&20.00\\
2015&2015&0.00&2.15&33.33\\

\bottomrule
\end{tabular} 
\\
\justify
%Note:
\end{table}

\begin{table}[H]
\centering\footnotesize
\caption{Observed firm births given synthetic births (manufacturing)} \label{ProbabilityManufacturing} \medskip
\renewcommand{\arraystretch}{1}
\begin{tabular}{c c| c c c}
\toprule
\multicolumn{2}{c|}{\textbf{First (Birth) Year}} &  \multicolumn{3}{c}{\textbf{\% of Births over NAICS}}\\
\textbf{Synthetic}&\textbf{Actual}&\textbf{Minimum}&\textbf{Mean}&\textbf{Maximum}\\
\midrule
1991&1991&4.76&31.64&52.03\\
1992&1992&0.00&3.32&10.53\\
1993&1993&0.00&3.97&33.33\\
1994&1994&0.00&4.21&33.33\\
1995&1995&0.00&4.41&20.00\\
1996&1996&0.00&5.36&33.33\\
1997&1997&0.00&4.09&16.94\\
1998&1998&0.00&5.46&25.00\\
1999&1999&0.00&5.27&33.33\\
2000&2000&0.00&3.39&25.00\\
2001&2001&0.00&2.19&10.00\\
2002&2002&0.00&2.45&25.00\\
2003&2003&0.00&1.71&10.00\\
2004&2004&0.00&2.07&17.86\\
2005&2005&0.00&1.92&16.67\\
2006&2006&0.00&2.49&50.00\\
2007&2007&0.00&1.74&14.29\\
2008&2008&0.00&1.60&20.00\\
2009&2009&0.00&1.60&20.00\\
2010&2010&0.00&1.34&33.33\\
2011&2011&0.00&2.43&50.00\\
2012&2012&0.00&1.93&20.00\\
2013&2013&0.00&1.61&20.00\\
2014&2014&0.00&1.71&14.29\\
2015&2015&0.00&1.41&14.29\\

\bottomrule
\end{tabular} 
\\
\justify
%Note:
\end{table}

\begin{figure} [H]
\centering
\caption{The difference between first and last year given synthetic first year} \label{SyntheticFirstYear}
\includegraphics[height=2.8in, width=.7\linewidth]{graphs/The_difference_between_first_and_last_year_given_synthetic_first_year_bw.pdf} 
\begin{minipage}{0.85\textwidth}
%{\footnotesize Note:  \par}
\end{minipage}
\end{figure}

\section{Conclusion and Extensions}

%Statistics Canada disseminates business data in highly aggregated forms. To get access to Canadian micro business databases, 
In this paper, we adapt and implement algorithms used to create the U.S. synthetic data for LBD to create Canadian synthetic data for LEAP. We show the newly created data set is analytically valid for a wide range of statistical analyses,  as well as provide evidence on confidentiality properties.

\subsection{Addition of variables that are not analytically valid}\todo{BD: ?}

\subsection{Addition of analytically valid variables}\todo{BD: ?}

Capital stock or revenue for incorporated \todo{BD: ?}

\todo{JA: I think we do not need these two sections.}

\newpage

\appendix{Supplementary Graphs}

\section{Analytical validity}

\subsection{Confidence interval for gross employment and other measures}
We compute the standard error for gross employment as follows. We consider gross employment $E$ to be the sum of firm employments $E_j$:

\begin{equation}
E = \sum_j E_j
\end{equation}

Average firm employment $\bar{E} = \frac{E}{N_j}$ is assumed to be normally distributed, with standard deviation $\sigma_{\bar{E}}$. We compare the synthetic and the confidential data for gross employment, including error bands.

\subsection{Confidence interval overlap measures}

More generally, the question as to the statistical precision of the results obtained from the synthetic data can be assessed. For this purpose, we computed the overlap of parameter estimates  as suggested by \cite{tas2006}. We compute the \emph{interval overlap measure} $J_{k,m}$ for parameter $k$ in model $m$. Consider the overlap of confidence intervals $(L,U)$ for $\beta_{k,m}$ (estimated from the confidential data) and $(L^{*},U^{*})$ for $\beta_{k,m}^*$ (from the synthetic data). Let $L^{over} = \max (L,L^{*} )$ and $U^{over} = \min (U,U^{*})$. Then the average overlap in confidence intervals is
$$
J_{k,m}^{*} = \frac{1}{2} \left [ \frac{U^{over} - L^{over}}{U-L} + \frac{U^{over} - L^{over}}{U^*-L ^*}        \right ]
$$
We then average $J_{k,m}^{*}$ over all estimated models and parameters, by validation request. The correct counterfactual involved running these validation requests against synthetic data that does not claim analytical validity, such as synthetic data generated from uni-dimensional distributions of variables. Results are pending.\todo{BD: ?}

%\subsection{Other models}
%
%Possible papers:
%
%\begin{itemize}
%%\item %\textcite{10.1257/aer.20141280} use the BDS to show the role of firm size in firm dynamics, but also had access to the Synthetic LBD.
%\item \textcite{NBERc0480} use a cross-country dataset to study average post-entry behavior of young firms. 
%\end{itemize}

%\bibliographystyle{apalike}
%\bibliography{paper}

\printbibliography

\end{document}