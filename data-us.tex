The \ac{LBD} \citep{LBD} is created from the U.S. Census Bureau's \ac{BR} by creating longitudinal links of establishments using name and address matching. The database has information on birth, death, location, industry,  firm  affiliation of employer establishments, and ownership by multi-establishment firms, as well as their employment over time, for nearly all sectors of the economy from 1976 through 2015 (as of this writing). It serves as a key linkage file as well as a research dataset in its own right for numerous research articles, as well as a tabulation input to the U.S. Census Bureau's \acl{BDS} \citep[\acs{BDS}]{BDS}. Other statistics created from the underlying Business Register include the \acl{CBP} \citep[\acs{CBP}]{CBP} and the \acl{SBUSB} \citep[\acs{SBUSB}]{SBUSB}. For a full description, readers should consult  \citet{RePEc:cen:wpaper:02-17,}. The key variables of interest for this experiment are birth and death dates, payroll, employment, and the industry coding of the establishment.  \citet{SJIAOS-2014d} explore a possible expansion of the synthesis methods described later to include location and firm affiliation. Note that information on payroll and employment does not come from individual-level wage records, as is the case for both the Canadian and German datasets described below, as well as for the \acl{QWI} \citep{AbowdEtAl2009} derived from the \acl{LEHD} \citep[LEHD]{RePEc:cen:wpaper:18-27} in the United States. \todo{Lars, can you include a sentence where the data come from, then?} Thus, methods that connect establishments based on labor flows \citep{BenedettoEtAl2007,RePEc:iab:iabfme:201006_en} are not employed. We also note that payroll is the cumulative sum of wages paid over the entire calendar year, whereas employment is measured as of March 12 of each year.


%The US LBD was created in the early 2000s (Miranda and Jarmin, 2002), following previous research files with more restrictive coverage. At its core, it is a research database containing longitudinally linked data records from a statistical business register of establishments. Breaks of longitudinal links are resolved using probabilistic name and address matching. The variables currently in the LBD are industry, annual payroll, employment, geography, birth year, death year, and firm structure. Though it has very few variables on the database itself, it serves as a backbone for many linkages into establishment and firm surveys and censuses such as business dynamics and job flows. 

%The fundamental structure of the LBD (and thus the SynLBD) is a longitudinal file on economic entities, where each entity has a start and end date and a small number of key attributes that evolve over time. Hypothetically, this structure is shared by many other longitudinal panels, such as panels of jobs or of residences. We should note that it does not apply to data structures like a linked employer-employee database, since there are no linkages between entities at a point in time. Thus, using concepts from graph theory, it is a mapping of a network that contains only nodes, and no edges. These structural characteristics are relevant for any attempt to generalize the synthesizing methodology to other contexts, such as matched employer-employee data (but see Barrientos et al., 2017).

%The primary goal of the SynLBD project is to create partially synthetic microdata on establishments for public release, allowing researchers easier access for the implementation of a wide (unconstrained) range of models with analytically valid inferences about the underlying population, while protecting against re-identification of any given unit or its attributes. There are multiple reasons why a public release of such data is desirable. The US LBD is one of most requested datasets in the Federal Statistical Research Data Centers (FSRDCs), but access through FSRDC the is still subject to long approval processes. In many European countries, access to data on business registers is arduous or impossible for researchers. Access through commercial providers is possible (Bureau Van Dijk), but coverage is generally poor.


