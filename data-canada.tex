
The \ac{LEAP} \citep{StatisticsCanada2019} contains information on annual employment for each employer business in all sectors of the Canadian economy. It covers incorporated and unincorporated businesses that issue at least one annual statement of remuneration paid (T4 slips) in any given calendar year. It excludes self-employed individuals or partnerships with non-salaried participants.

To construct the \ac{LEAP}, Statistics Canada uses three sources of information: (1) T4 administrative data  from the Canada Revenue Agency (CRA), (2) data from Statistics Canada's \acl{BR} \citep{StatisticsCanada2019a}, and (3) data from  Statistics Canada's \acf{SEPH} \citep{StatisticsCanada2019b}. 



%\begin{description}
%\item[T4] 
In general, all employers in Canada provide employees with a T4 slip if they paid employment income, taxable allowances and benefits, or any other remuneration in any calendar year. The T4 information is reported to the tax agency, which in turn provides this information to Statistics Canada. 

%\item[BR] 
The Business Register is Statistics Canada's central repository of baseline information on businesses and institutions operating in Canada. It is used as the survey frame for all business related data sets.

%\item[SEPH] 
The objective of the \ac{SEPH} is to provide monthly information on the level of earnings, the number of jobs, and hours worked by detailed industry at the national and provincial levels. To do so, it combines a census of approximately one million payroll deductions provided by the CRA, and the Business Payrolls Survey, a sample of 15,000 establishments.  
%\end{description}

The core \ac{LEAP}  contains four variables (1) a longitudinal Business Register Identifier (LBRID), (2) an industry classification, (3) payroll and (4) a measure of employment. 

%\begin{enumerate}

%\item 
The LBRID uniquely identifies each enterprise and is derived from the Business Register. To avoid ``false'' deaths and births due to mergers, restructuring or changes in reporting practices, Statistics Canada uses employment flows. Similar to \citet{BenedettoEtAl2007} and \citet{RePEc:iab:iabfme:201006_en}, the method  compares cluster of workers in each newly identified enterprise with all the clusters of workers in firms from the previous year. This comparison yields a new identifier (LBRID) derived from those of the \ac{BR}.

%\item 
The industry classification comes from the \ac{BR} for single-industry firms. If a firm operates in multiple industries, information on payroll from the \ac{SEPH} is used to identify the industry in which the firm pays the highest payroll. Prior to 1991, information on industry was based on the SIC,  but it is currently based on the  North American Industrial Classification System (NAICS). We use the information at the NAICS four-digit (industry group) level. 

%\item 
The firm's payroll is measured as the sum of all T4s  reported to the CRA for the calendar year.

%\item 
Employment is measured either using \ac{ILU} or \ac{ALU}. \acp{ALU} are obtained by dividing the payroll by the average annual earnings in its industry/province/class category computed using the \ac{SEPH}. \acp{ILU} are a head count of the number of T4 issued by the enterprise, with employees working for multiple employers split proportionately across firms according to their total annual payroll earned in each firm. 


%\end{enumerate}
% Commented out: too much an advertisement... jars with the rest.
% The \ac{LEAP} is the only data set in Canada  that allows research on a variety of themes, like employment growth, industry turnover, firm survival, job creation and job destruction, etc. 

For the purpose of this experiment,  we exclude the public sector (NAICS 61, 62, and 91), even though they are contained in the database, because they may not be accurately captured \citep{StatisticsCanada2019}. Statistics Canada does not publish any statistics for those sectors.