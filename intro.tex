There has been growing interest in the creation and use of synthetic data in order to make confidential micro data easier to use and access for researchers. Synthetic data are created by replacing sensitive values with repeated draws from a model fit to the original data (Little, 1993; Rubin, 1993). The approach is closely related to multiple imputations.

%Drechsler (CITE) and \cite{RePEc:bla:istatr:v:79:y:2011:i:3:p:362-384} %have proposed solutions based on the creation of synthetic business microdata.\footnote{} In the latter case, access to the synthetic data is combined with a validation server (CITES). 

\todo{Discuss the advantages and inconveniences of creating and using synthetic data compared to other ways of making the data more accessible}

With respect to micro-level firm data, the focus of this article, one of the most well-known and successful initiative is the U.S. Longitudinal Business Database (LBD).  

\todo{describe the LBD in more details} 

In this paper, we discuss two new initiatives in Germany and Canada, that builds on the methods developed for the LBD to create similar micro-level data sets at the firm level. 

In Canada, access to micro-level business data is done through the Canadian Center for Data Development and Economic Research (CDER) located in the headquarters of Statistics Canada in Ottawa. This means that researchers who need access to micro-level business data have to be physically present in Ottawa in order to do research, a major limitation on the use of those data holdings.

In Germany, \todo{Discuss German case}

We verify the analytical validity of synthetic data files so created along a variety of measures. First, we show that more average firm characteristics (gross employment, total payroll) in the synthetic data closely match those from the original data. Second, we also find the synthetic data close replicates various measures of firm dynamics (entry and exit rates) and job flows (gross and net job creation rate) from the original data. Finally, we assess whether measures of economic growth vary between both data sets using a dynamic panel data models and find that both data sets yield similar predictions.

In each case, to provide evidence on the confidentiality properties this newly created  synthetic database, we estimate the probability that the synthetic first year equals the true first year given the synthetic fist year and find that those probabilities are quite low except for the first year of LEAP database. The probability for the first year is higher because of censoring and lack of previous information.

%The rest of the paper is organized as follows. Section 2 provides a detailed description of the LEAP database and its uses. Section 3 defines what we mean by synthetic data and describe how we created the synthetic LEAP. In section 4, the analytical validity of synthetic database is assessed. Section 5 concludes.
