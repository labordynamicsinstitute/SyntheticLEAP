Establishment and firm microdata pose many challenges to the application of disclosure avoidance techniques and thus to public dissemination, as they are sparse in certain dimensions (e.g. detailed industry and geography) and often unique. Moreover, the distribution of business data are often highly skewed: it is easy to concoct examples of firms and establishments that are so dominant in their industry or location that they would be immediately identified if their data were publicly released. Finally, there are also greater financial incentives to identifying the particulars of some firms and their competitors. This is true for many countries. 

Consequently, it is not uncommon that access to establishment microdata, if granted at all, is provided through data enclaves (Research Data Centers), at headquarters of statistical agencies, or some other limited means, under strict security conditions. These restrictions on data access reduce the growth of knowledge by increasing the cost to researchers of accessing the data.

The use of synthetic data as either a replacement for traditional public-use data or as a mechanism to provide secure access to confidential data remains novel to most social scientists, despite several datasets having been made available. Synthetic data are created by replacing sensitive values with repeated draws from a model fit to the original data (Little, 1993; Rubin, 1993), in an approach that is closely related to multiple imputation.

By making disclosable synthetic microdata available through a remotely accessible data server, combined with a validation server, the SynLBD approach alleviates some of the access restrictions associated with economic data. The approach is mutually beneficial to both agency and researchers. Researchers can access public use servers at little or no cost within a few weeks of their initial application and can later validate their model-based inferences on the full confidential microdata.

The most well-know and successful effort to use synthetic data in this fashion is the U.S. Longitudinal Business Database (LBD). \todo{describe the LBD in more details. Summarize below?} 

%The US LBD was created in the early 2000s (Miranda and Jarmin, 2002), following previous research files with more restrictive coverage. At its core, it is a research database containing longitudinally linked data records from a statistical business register of establishments. Breaks of longitudinal links are resolved using probabilistic name and address matching. The variables currently in the LBD are industry, annual payroll, employment, geography, birth year, death year, and firm structure. Though it has very few variables on the database itself, it serves as a backbone for many linkages into establishment and firm surveys and censuses such as business dynamics and job flows. 

%The fundamental structure of the LBD (and thus the SynLBD) is a longitudinal file on economic entities, where each entity has a start and end date and a small number of key attributes that evolve over time. Hypothetically, this structure is shared by many other longitudinal panels, such as panels of jobs or of residences. We should note that it does not apply to data structures like a linked employer-employee database, since there are no linkages between entities at a point in time. Thus, using concepts from graph theory, it is a mapping of a network that contains only nodes, and no edges. These structural characteristics are relevant for any attempt to generalize the synthesizing methodology to other contexts, such as matched employer-employee data (but see Barrientos et al., 2017).

%The primary goal of the SynLBD project is to create partially synthetic microdata on establishments for public release, allowing researchers easier access for the implementation of a wide (unconstrained) range of models with analytically valid inferences about the underlying population, while protecting against re-identification of any given unit or its attributes. There are multiple reasons why a public release of such data is desirable. The US LBD is one of most requested datasets in the Federal Statistical Research Data Centers (FSRDCs), but access through FSRDC the is still subject to long approval processes. In many European countries, access to data on business registers is arduous or impossible for researchers. Access through commercial providers is possible (Bureau Van Dijk), but coverage is generally poor.

%The US SynLBD was released in 2010 to the Cornell SDS. The Census Bureau’s Disclosure Review Board (DRB), as well as the Internal Revenue Service (IRS), classified SynLBD as public-use, but access is controlled due to concerns about the quality of the data. There are no disclosure concerns but researchers are cautioned not to trust results as if they were created by a traditional public-use file without going through the validation process. For similar reasons, the preparation of tabular data based on the synthetic data is strongly discouraged, and are not validated. Nevertheless, the synthetic data are of much easier access than the confidential data.

%References

%Miranda, J. and Jarmin, R. (2002). The Longitudinal Business Database. Discussion Paper CES-WP-02-17, U.S. Census Bureau, Center for Economic Studies.

%Barrientos, A. F., Bolton, A., Balmat, T., Reiter, J. P., de Figueiredo, J. M., Machanava- jjhala, A., Chen, Y., Kneifel, C., and DeLong, M. (2017). A framework for sharing confidential research data, applied to investigating differential pay by race in the u. s. government. Technical Report 1705.07872v1, arXiv.

In this paper, we discuss two new initiatives in Germany and Canada, that builds on the methods developed for the US SynLBD to create similar synthetic micro firm-level data sets. 

In Canada, the Canadian Center for Data Development and Economic Research (CDER) was created in 2011 to allow Statistics Canada to make better use of its business data holdings, without compromising security. Secure access to to business microdata for approved analytical research projects is done through a physical facility located in Statistics Canada’s headquarters. 

CDER implements many risks mitigation measures to alleviate the security risks specific to micro-level business data including limits on tabular outputs, centralized vetting, monitoring of programs logs. Access to the data is done through a Statistics Canada designed interface in which actual observations cannot be viewed. But the most significant barrier to access is the travel cost of coming to Ottawa.

In Germany, \todo{Discuss German case}

\todo{do modify according to what was exactly done in the respective cases of Canada and Germany}

We verify the analytical validity of synthetic data files so created along a variety of measures. First, we show that more average firm characteristics (gross employment, total payroll) in the synthetic data closely match those from the original data. Second, we also find the synthetic data close replicates various measures of firm dynamics (entry and exit rates) and job flows (gross and net job creation rate) from the original data. Finally, we assess whether measures of economic growth vary between both data sets using a dynamic panel data models and find that both data sets yield similar predictions.

In each case, to provide evidence on the confidentiality properties this newly created  synthetic database, we estimate the probability that the synthetic first year equals the true first year given the synthetic fist year and find that those probabilities are quite low except for the first year of LEAP database. The probability for the first year is higher because of censoring and lack of previous information.

%The rest of the paper is organized as follows. Section 2 provides a detailed description of the LEAP database and its uses. Section 3 defines what we mean by synthetic data and describe how we created the synthetic LEAP. In section 4, the analytical validity of synthetic database is assessed. Section 5 concludes.
