While Canada maintains a network of research data centers similar to the one in the U.S., there is virtually no firm-level data sets amongst the data holdings of the Canadian Research Data Center Networks (CRDCN). Researchers who need access to confidential micro-level business data have to go to the Canadian Center for Data Development and Economic Research (CDER) located in the headquarters of Statistics Canada. One reason commonly heard for this limited access has to do with the fact that Canada is a small country with a highly skewed distribution of firms, thus compounding problems linked to confidentiality protection. 

Confidentiality protection with CDER is done through a variety of means, including remote execution and research monitoring. But more importantly, researchers accessing CDER data holdings must become \textit{deemed employees} of Statistics Canada and sign an asset-freeze agreement to ensure they do not profit financially from their research. Since 2018, Statistics Canada has undertaken a so-called Modernization initiative. One of the objective of that Initiative is to improve access to researchers to confidential firm-level micro data.

The situation described above for access to Canadian business microdata is not unique. Drechsler (CITE) and \cite{RePEc:bla:istatr:v:79:y:2011:i:3:p:362-384} have proposed solutions based on the creation of synthetic business microdata.\footnote{Synthetic data are created by replacing sensitive values with repeated draws from a model fit to the original data (Little, 1993; Rubin, 1993). The approach is closely related to multiple imputations.} In the latter case, access to the synthetic data is combined with a validation server (CITES). 

To reduce the cost of granting and obtaining access to the Canadian business microdata, we created  synthetic data for the  \ac{LEAP} database. In this project, we have three objectives: i) evaluate to what extent synthetic code developed for the U.S. Longitudinal Business Database (LBD) can easily transferable to LEAP database with comparable structure; ii) examine whether the automated synthesis will generate useful datasets that offer analytical validity for a wide range of statistical analyses; iii) provide evidence on confidentiality properties. 


We verify the analytical validity of synthetic data set so created along a variety of measures. First, we show that more average firm characteristics (gross employment, total payroll) in the synthetic data closely match those from the original data. Second, we also find the synthetic data close replicates various measures of firm dynamics (entry and exit rates) and job flows (gross and net job creation rate) from the original data. Finally, we assess whether measures of economic growth vary between both data sets using a dynamic panel data models and find that both data sets yield similar predictions.

To provide evidence on the confidentiality properties this newly created  synthetic database, we estimate the probability that the synthetic first year equals the true first year given the synthetic fist year and find that those probabilities are quite low except for the first year of LEAP database. The probability for the first year is higher because of censoring and lack of previous information.\todo{BD: I have no idea what this means, please explain better.} \todo{JA: I do not know what else I could add here.}

The rest of the paper is organized as follows. Section 2 provides a detailed description of the LEAP database and its uses. Section 3 defines what we mean by synthetic data and describe how we created the synthetic LEAP. In section 4, the analytical validity of synthetic database is assessed. Section 5 concludes.
