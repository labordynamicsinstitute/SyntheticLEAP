\documentclass[10pt,twoside]{article}
\usepackage[paperheight=238mm,paperwidth=166mm,text={128mm,198mm},centering]{geometry}
\usepackage{amssymb,amsmath}
\usepackage{scrextend}
\usepackage{setspace}
\usepackage{lipsum}
\usepackage{titlesec}
%\usepackage[T1]{fontenc}

\usepackage{helvet}
\renewcommand{\familydefault}{\sfdefault}

\usepackage{mathptmx} 
\usepackage{fancyhdr}

\def \sitissue{STATISTICS IN TRANSITION new series, March 2018}
\def \sitshort{First A., Second A.: A Simple Manuscript ...  }

\pagestyle{fancy}
\fancyhead{}
\fancyhead[LO]{{\small \textit{\sitissue}}}
\fancyhead[RO]{{\small \textit{\thepage}}}
\fancyhead[RE]{{\small \textit{\sitshort}}}
\fancyhead[LE]{{\small \textit{\thepage}}}
\fancyfoot[CE]{}
\fancyfoot[CO]{}

\linespread{1.07} %Set 1.3 for interline 1.5 

\titleformat{\section}
{\normalfont\bfseries\large}{\thesection.}{2mm}{}

\titleformat{\subsection}
{\normalfont\bfseries}{\thesubsection.}{2mm}{}

%%%%%%%%%%%%%%%%%%%%%%%%%%%%%%%%%%%%%%%%%%%%%%%%
%%%%%%%%%%%%%%%%%%%%%%%%%%%%%%%%%%%%%%%%%%%%%%%%
%%%%%%%%%%%%%%%%%%%%%%%%%%%%%%%%%%%%%%%%%%%%%%%%

\begin{document}
\vspace*{-8mm}
\begin{small}
\noindent {\em \sitissue }\\
{\em Vol. XX, No. X, pp. xxx--xxx}
\end{small}

\begin{onehalfspace}
	\begin{center} 
		{\Large \bf  A SIMPLE MANUSCRIPT TEMPLATE \\ FOR THE STATISTICS IN TRANSITION JOURNAL} 
	\end{center}
\end{onehalfspace}

\vspace*{-4mm}

\begin{center}
	\begin{large} 
	{\bf First Author}\footnote{Institutional affiliation. E-mail: some@one.edu},
	{\bf Second Author}\footnote{Institutional affiliation. E-mail: author2@address.com}	 
    \end{large}
\end{center}


\begin{center}\begin{large} \textbf{ABSTRACT} \end{large} \end{center}

\begin{addmargin}[6mm]{6mm}
\begin{small}
\begin{singlespace}

This is a simple template file facilitating the preparation of manuscripts for the Statistics in Transition journal using \LaTeX. Please enter your abstract text here.

\smallskip \noindent \textbf{Key words:} template, article, journal.

\end{singlespace}
\end{small}
\end{addmargin}

\smallskip

\section{The first section}\label{sec1}

The first section starts here. 

\subsection{Citations}

Refer to the literature using Harvard convention, as exemplified by the following articles of Gamrot (2012, 2013), a proceedings paper of Kennickell (1997) or a book of S\"{a}rndal, Swensson and Wretman (1992).

\subsection{Formulae}

Expressions should not exceed the text width and may be entered as follows:

\begin{equation}
V(\hat{t})=\sum_{i,j \in U} \check{y}_i \check{y}_j \Delta_{ij}
\end{equation}

\begin{equation}
\hat{V}(\hat{t})=\sum_{i,j \in s} \check{y}_i \check{y}_j \frac{\Delta_{ij}}{\pi_{ij}}
\end{equation}


\subsection{Tables}

An example table:


\begin{center}
	
	\textbf{Table 1.} Some nicely looking numbers\\\smallskip
\begin{small}
\begin{tabular}{ccccc}\hline
	$U$ & $W$ & $X$ & $Y$ & $Z$ \\\hline
	1001 & 1002 & 1003 & 1004 & 1005 \\
	1006 & 1007 & 1008 & 1009 & 1000 \\
	1001 & 1002 & 1003 & 1004 & 1005 \\
	1006 & 1007 & 1008 & 1009 & 1000 \\\hline	
\end{tabular}
\end{small}

\end{center}	






\section{The second section}

\lipsum[1]

\section{Conclusions}

\lipsum[1]

\section*{Acknowledgements}

Thanks to anyone for support, funding and such may be included in the non-numbered Acknowledgements section.


\noindent \begin{center}\begin{large}{\bf REFERENCES}\end{large}\end{center}

\everypar = {
\parindent=0pt
\hangindent=8mm
\hangafter=1
}\noindent 

GAMROT, W., (2012). Estimation of Finite Population Kurtosis under Two-Phase Sampling for Nonresponse. \textit{Statistical Papers}, 53, 887--894.

GAMROT, W., (2013). Maximum Likelihood Estimation for Ordered Expectations of Correlated Binary Variables. \textit{Statistical Papers}, 54, 727--739.

KENNICKELL, A. B. (1997). Multiple Imputation and Disclosure Protection:
The Case of the 1995 Survey of Consumer Finances. In \textit{Record Linkage
Techniques}. W. Alvey and B. Jamerson (eds.) Washington D. C.: National
Academy Press, 248--267.

S\"{A}RNDAL, C-E., SWENSSON, B., WRETMAN, J., (1992). \textit{Model Assisted
Survey Sampling}, New York: Springer.


\end{document}
